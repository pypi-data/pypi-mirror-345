\documentclass{galahad}

% set the package name

\newcommand{\package}{lanb}
\newcommand{\packagename}{LANCELOT B}
\newcommand{\fullpackagename}{LANC\-E\-LOT}
\newcommand{\solver}{{\tt \fullpackagename\_solve}}

% other local definitions

\newcommand{\calE}{{\cal E}}
\newcommand{\calG}{{\cal G}}

\begin{document}

\galheader

%%%%%%%%%%%%%%%%%%%%%% SUMMARY %%%%%%%%%%%%%%%%%%%%%%

\galsummary

{\tt \packagename} is a suite of Fortran procedures for minimizing an
objective function, where the minimization variables are required to
satisfy a set of auxiliary, possibly nonlinear, constraints.  The
objective function is represented as a sum of ``group'' functions.  Each
group function may be a linear or nonlinear functional whose argument is
the sum of ``finite element'' functions; each finite element function is
assumed to involve only a few variables or have a second derivative
matrix with low rank for other reasons. The constraints are also
represented as group functions.  Bounds on the variables and known
values may be specified.  The routine is especially effective on
problems involving a large number of variables.

\galbasics

The objective function has the (so-called {\em group-partial separable}) form
\eqn{objective}{f(\bmx) = \sum_{i \in \calG_O} w^g_i g_i \left(
       \sum_{j \in \calE_i} w^e_{ij} e_j(\bmx^e_j, \bmp^e_j)
  + \bma_i^T \bmx - b_i, \bmp^g_i \right),
\; \; \; \; \; \bmx = ( x_1, x_2, \ldots, x_n)^T,}
where
$\calG_O$ is a subset of $\calG=\{1,  \ldots, n_g \}$,  the list of indices
of group functions $g_i$, each
$\calE_i$ is a subset of $\calE=\{1,  \ldots, n_e \}$,  the list of
indices of nonlinear element functions $e_j$, and
the $w^g_i$, $\bmp^g_i$, $w^e_{ij}$ and $\bmp^e_{i}$
are, respectively, group and element weights and vectors of parameters.
Furthermore, the  vectors $\bmx^e_j$  are either small subsets of the
minimization variables $\bmx$
or are such that the rank of the second derivative matrix
of the nonlinear element $e_j$ is small for some other reason, and the
vectors $\bma_i$ of the linear elements  are sparse.  The  least value of
the objective function within the intersection of
the regions defined by the general constraints
\eqn{constraints}{c_i(\bmx) = w^g_i g_i \left(
       \sum_{j \in \calE_i} w^e_{ij} e_j(\bmx^e_j, \bmp^e_j)
  + \bma_i^T \bmx - b_i, \bmp^g_i \right) = 0, \; \; \; \; \;
 i \in \calG_C = \calG \; \backslash \; \calG_O \; \backslash \; \calG_I,}
and the ``box''
\disp{l_j \leq x_j \leq u_j, \; \; \; \; 1 \leq j \leq n,}
is sough; here $\calG_I$ is another subset of  $\calG$ that may be used to
specify a list of groups to be (temporarily) ignored.
Either bound on each variable may be infinite, and special
considerations are taken when there are no general constraints.

The method used is iterative. There are two levels of iteration. In
the outer, a composite function, the (augmented Lagrangian) merit function,
\eqn{objectiveb}{\phi( \bmx, \bmy, \mu ) = f(\bmx) + \sum_{i \in \calG_C}
 y_i c_i (\bmx) + \frac{1}{2 \mu} \sum_{i \in \calG_C} (c_i (\bmx))^2,}
where $\mu$ is known as the penalty parameter and $\bmy$ is a vector of
Lagrange multiplier estimates,
is formulated. Each outer iteration requires the approximate
minimization of this merit function
within the feasible box, for given values of $\mu$ and $\bmy$.

The required approximate minimization for fixed  $\mu$ and $\bmy$ is
carried out using a series of inner iterations.  At each inner
iteration, a quadratic model of the merit function is
constructed.  An approximation to the minimizer of this model within a
trust-region is calculated.  The trust region can be either a ``box'' or
a hypersphere of specified radius, centered at the current best estimate
of the minimizer.  If there is an accurate agreement between the model
and the true objective function at the new approximation to the
minimizer, this approximation becomes the new best estimate.  Otherwise,
the radius of the trust region is reduced and a new approximate
minimizer sought.  The algorithm also allows the trust-region radius to
increase when necessary. The minimization of the model function can be
carried out by using either a direct-matrix or an iterative approach.

The approximate minimization of the model is performed in two stages.
In the first, a so-called generalized Cauchy point is determined
by approximately minimizing the model within the intersection of
the feasible box and the trust-region along a scaled steepest descent
direction. Having taken this step, the model is further reduced
by solving one or more quadratic minimization problems in which
any constraints activated at the Cauchy point remain so. The latter
computation is essentially equivalent to the solution of a sequence
of linear systems, and may be performed using either direct (factorization)
or iterative (preconditioned conjugate gradient) method.

After an appropriate approximation to the minimizer of the merit
function is obtained, and if there are general constraints,
$\mu$ and $\bmy$ are adjusted to ensure
convergence of the outer iteration to the required solution of the
constrained minimization problem.

If there is a linear transformation of variables such that one or more
of the element functions depend on fewer transformed ``internal''
variables than the original number of variables $\bmx^e_i$, this may be
specified. This can lead to a substantial reduction in computing time.

There are facilities for the user to provide a preconditioner for the
conjugate gradient solution of the internal linear systems (a default is
provided) and to provide (optional) first and second derivatives for (subsets)
of the element
functions.  The subroutine optionally returns control to the user to calculate
function and derivative values of the group functions $g_i$ and nonlinear
element functions $e_j$. A large number of other options are available,
such as
the ability to choose between different generalized Cauchy points, to select
different trust-region shapes, including a structured trust region especially
designed for functions with structure like those in
\req{objective}--\req{constraints}, and to permit non-monotonic descent of
the merit function. Some of these options require external packages,
and thus are not available by default (see Section~\ref{galexternal}).

\galexternal

Some of the linear system options make use of the HSL package
{\tt MA27}. This package is available without charge for academic
purposes from the HSL archive
\begin{verbatim}
    http://hsl.rl.ac.uk/archive/hslarchive.html .
\end{verbatim}
Another option, the Lin-Mor\'{e} preconditioner {\tt ICFS}, is available as
part of the MINPACK 2 library, via
\begin{verbatim}
    http://www-unix.mcs.anl.gov/~more/icfs/ .
\end{verbatim}
Munksgaard's preconditioner {\tt MA61} is only available commercially
from HSL. See
\begin{verbatim}
    http://www.cse.clrc.ac.uk/Activity/HSL+152
\end{verbatim}
for details. In addition, the \galahad\ package {\tt SILS} may
be replaced by the functionally equivalent but state-of-the-art
HSL package {\tt HSL\_MA57} from the same source.

%%%%%%%%%%%%%%%%%%%%%% attributes %%%%%%%%%%%%%%%%%%%%%%

\galattributes
\galversions{\tt  \fullpackagename\_single, \fullpackagename\_double},
\galuses
{\tt GALAHAD\_CPU\_time},
{\tt GALAHAD\_SPECFILE},
{\tt LANCELOT\_INITW},
{\tt LANCELOT\_OTHERS},
{\tt LANCELOT\_HSPRD},
{\tt LANCELOT\_CAUCHY},
{\tt LANCELOT\_CG},
{\tt LANCELOT\_PRECN},
{\tt LANC\-ELOT\_FRNTL},
{\tt LANCELOT\_STRUTR},
{\tt GALAHAD\_SMT},
{\tt GAL\-AHAD\_SILS},
{\tt GALAHAD\_SCU},
{\tt LANCELOT\_ASMBL},
{\tt LANCELOT\_EXTEND}.
\galdate February 2002.
\galorigin N. I. M. Gould, Rutherford Appleton Laboratory, England, and
Ph. L. Toint, University of Namur, Belgium.
Previous Fortran 77 version also by A. R. Conn, IBM T. J.
Watson Research Center, USA.
\gallanguage Fortran~95 + TR 15581 or Fortran~2003.
\galremark This is an enhanced version of the Fortran 77 subroutines
{\tt SBMIN} and {\tt AUGLG},
available as part of {\sf LANCELOT A}, and supersedes them.

%%%%%%%%%%%%%%%%%%%%%% HOW TO USE %%%%%%%%%%%%%%%%%%%%%%

\galhowto

The package is available with single, double and (if available) 
quadruple precision reals, and either 32-bit or 64-bit integers. 
Access to the 32-bit integer,
single precision version requires the {\tt USE} statement
\medskip

\hspace{8mm} {\tt USE \fullpackagename\_single}

\medskip
\noindent
with the obvious substitution 
{\tt \fullpackagename\_double},
{\tt \fullpackagename\_quadruple},
{\tt \fullpackagename\_single\_64},
{\tt \fullpackagename\_double\_64} and
{\tt \fullpackagename\_quadruple\_64} for the other variants.


\noindent
If it is required to use more than one of the modules at the same time, 
the derived types
{\tt \fullpackagename\_problem\_type},
{\tt \fullpackagename\_\-time\_type},
{\tt \fullpackagename\_control\_type},
{\tt \fullpackagename\_inform\_type}
(Section~\ref{galtypes})
and the subroutines
{\tt \fullpackagename\_init\-ialize},
{\tt \fullpackagename\_\-solve},
{\tt \fullpackagename\_terminate},
(Section~\ref{galarguments})
and
{\tt \fullpackagename\_read\_specfile}
(Section~\ref{galfeatures})
must be renamed on one of the {\tt USE} statements.

%%%%%%%%%%%%%%%%%%%%%%%%%%% kinds %%%%%%%%%%%%%%%%%%%%%%%%%%%

%%%%%%%%%%%%%%%%%%%%%% kinds %%%%%%%%%%%%%%%%%%%%%%

\galkinds
We use the terms integer and real to refer to the fortran
keywords {\tt REAL(rp\_)} and {\tt INTEGER(ip\_)}, where 
{\tt rp\_} and {\tt ip\_} are the relevant kind values for the
real and integer types employed by the particular module in use. 
The former are equivalent to
default {\tt REAL} for the single precision versions,
{\tt DOUBLE PRECISION} for the double precision cases
and quadruple-precision if 128-bit reals are available, and 
correspond to {\tt rp\_ = real32}, {\tt rp\_ = real64} and
{\tt rp\_ = real128} respectively as defined by the 
fortran {\tt iso\_fortran\_env} module.
The latter are default (32-bit) and long (64-bit) integers, and
correspond to {\tt ip\_ = int32} and {\tt ip\_ = int64},
respectively, again from the {\tt iso\_fortran\_env} module.


%%%%%%%%%%%%%%%%%%%%%% derived types %%%%%%%%%%%%%%%%%%%%%%

\galtypes
Five derived data types are accessible from the package.

%%%%%%%%%%% problem type %%%%%%%%%

\subsubsection{The derived data type for holding the problem}\label{typeprob}
The derived data type {\tt \fullpackagename\_problem\_type} is used to hold
the problem data. The components of
{\tt \fullpackagename\_problem\_type}
are:

\begin{description}

\ittf{n} is a scalar variable of type \integer, that holds
the number of optimization variables, $n$.

\ittf{ng} is a scalar variable of type \integer, that holds
the number of group functions, $n_g$.

\ittf{nel} is a scalar variable of type \integer, that holds
the number of nonlinear element functions, $n_{e}$.

\itt{IELING} is a rank-one allocatable array of dimension {\tt ISTADG(ng+1)-1}
and type \integer, that holds the indices of the nonlinear elements
$\calE_i$ used by each group.
The indices for group $i$ must immediately precede those for
group $i+1$ and each group's indices must appear in a contiguous
list.  See Section~\ref{galexample} for an example.

\itt{ISTADG} is a rank-one allocatable array of dimension {\tt ng+1} and
type \integer, whose $i$-th value gives the position in
{\tt IELING} of the first nonlinear element in group function $i$.  In
addition, {\tt ISTADG(ng+1)} should be equal to the position in
{\tt IELING} of the last nonlinear element in group {\tt ng} plus one. See
Table~\ref{2.1a} and Section~\ref{galexample} for an example.

\bctable{|c|c|c|c|c}
\cline{1-4}
elements & elements & \ldots & elements & {\tt IELING} \\
in $g_1$ & in $g_2$ &        & in $g_n$ &              \\
\cline{1-4}
weights of & weights of &    & weights of &            \\
elements & elements & \hspace*{4mm} \ldots \hspace*{4mm} & elements & {\tt ESCALE} \\
in $g_1$ & in $g_2$ &        & in $g_n$ &              \\
\cline{1-4}
\multicolumn{1}{l}{\hspace*{-2mm}$\uparrow$} &
\multicolumn{1}{l}{\hspace*{-2mm}$\uparrow$} &
\multicolumn{1}{l}{\hspace*{-2mm}$\uparrow$} &
\multicolumn{1}{l}{\hspace*{-2mm}$\uparrow$} &
\multicolumn{1}{l}{\hspace*{-2mm}$\uparrow$} \\
\multicolumn{1}{l}{\hspace*{-8mm}\tt ISTADG(1)} &
\multicolumn{1}{l}{\hspace*{-8mm}\tt ISTADG(2)} &
\multicolumn{1}{l}{\hspace*{-8mm}\tt ISTADG(3)} &
\multicolumn{1}{l}{\hspace*{-8mm}\tt ISTADG($n_g$)} &
\multicolumn{1}{l}{\hspace*{-8mm}\tt ISTADG($n_g+1$)}
\ectable{\label{2.1a}Contents of the arrays {\tt IELING}, {\tt ESCALE}
and {\tt ISTADG}}
\vspace*{-4mm}

\itt{IELVAR} is a rank-one allocatable array of dimension {\tt ISTAEV(nel+1)-1}
and type  \integer, that holds
the indices of the variables in the first nonlinear element $e_1$,
followed by those in the second nonlinear element $e_2$,
\ldots.   See Section~\ref{galexample} for an example.

\itt{ISTAEV} is a rank-one allocatable array of dimension {\tt nel+1} and type
\integer, whose $k$-th value is the position of the first variable
of the $k$-th nonlinear element function,
in the list {\tt IELVAR}. In addition, {\tt ISTAEV($n_e$+1)}
must be equal to the position of the last variable of element
$n_e$ in {\tt IELVAR}
plus one. See Table~\ref{2.2a} and
Section~\ref{galexample} for an example.

\bctable{|c|c|c|c|c}
\cline{1-4}
variables      & variables & \hspace*{5mm} \ldots \hspace*{5mm} & variables
   & {\tt IELVAR} \\
in $\bmx^e_1$ & in $\bmx^e_2$
&  & in $\bmx^e_n$ &              \\
\cline{1-4}
\multicolumn{1}{l}{\hspace*{-2mm}$\uparrow$} &
\multicolumn{1}{l}{\hspace*{-2mm}$\uparrow$} &
\multicolumn{1}{l}{\hspace*{-2mm}$\uparrow$} &
\multicolumn{1}{l}{\hspace*{-2mm}$\uparrow$} &
\multicolumn{1}{l}{\hspace*{-2mm}$\uparrow$} \\
\multicolumn{1}{l}{\hspace*{-8mm}\tt ISTAEV(1)} &
\multicolumn{1}{l}{\hspace*{-8mm}\tt ISTAEV(2)} &
\multicolumn{1}{l}{\hspace*{-8mm}\tt ISTAEV(3)} &
\multicolumn{1}{l}{\hspace*{-9mm}\tt ISTAEV($n_e$)} &
\multicolumn{1}{l}{\hspace*{-7mm}\tt ISTAEV($n_e+1$)}
\ectable{\label{2.2a}Contents of the arrays {\tt IELVAR} and {\tt ISTAEV}}
\vspace*{-4mm}

\itt{INTVAR} is a rank-one allocatable array of dimension {\tt nel+1} and
type \integer, whose $i$-th value must be set to the number of
internal variables required for the $i$-th nonlinear element function
$e_i$ on initial entry.  See Section~\ref{internal_and_ranges} for the
precise definition of internal variables and Section~\ref{galexample}
for an example.

\itt{ISTADH} is a rank-one allocatable array of dimension {\tt nel+1} and
type \integer, that is set in \solver\ so that its $i$-th value
$(1 \leq i \leq n_e )$ gives the position in the array {\tt FUVALS} (see
Sections~\ref{min} and \ref{efv}, and (particularly)
Tables~\ref{2.4a} and \ref{2.4b}) of
the first component of the Hessian matrix of the $i$-th nonlinear
element function $e_i$, {\em with respect to its internal variables}.
Only the upper triangular part of each Hessian matrix is stored and the
storage is by columns (see Section~\ref{reverse} for details).  The
element {\tt ISTADH($n_e +1$)} gives the position in {\tt FUVALS} of the
first component of the gradient of the objective function.

\itt{ICNA} is a rank-one allocatable array of dimension {\tt ISTADA(ng+1)-1}
and type \integer, that holds the indices of the
nonzero components of $a_1$, the gradient of the first linear element,
in any order, followed by those in $a_2$, etc.
See Table~\ref{2.3a} and Section~\ref{galexample} for an example.

\itt{ISTADA} is a rank-one allocatable array of dimension {\tt ng+1} and type
\integer, whose $i$-th
value is the position  of the  first nonzero component of
the $i$-th linear element gradient, $a_i$,
in the list {\tt ICNA}. In addition,
{\tt ISTADA(ng+1)}
must be equal to the position of the last nonzero component of
$a_{n_g}$ in {\tt ICNA}
plus one.
See Table~\ref{2.3a} and Section~\ref{galexample} for an example.

\bctable{|c|c|c|c|c}
\cline{1-4}
nonzeros  & nonzeros &  \hspace*{5mm} \ldots  \hspace*{5mm}   & nonzeros     &   {\tt A}   \\
in $\bma_1$  & in $\bma_2$ &             & in $\bma_{n_g}$ &             \\
\cline{1-4}
variables & variables &  \ldots    & variables    & {\tt ICNA}  \\
in $\bma_1$  & in $\bma_2$ &             & in $\bma_{n_g}$ &             \\
\cline{1-4}
\multicolumn{1}{l}{\hspace*{-2mm}$\uparrow$} &
\multicolumn{1}{l}{\hspace*{-2mm}$\uparrow$} &
\multicolumn{1}{l}{\hspace*{-2mm}$\uparrow$} &
\multicolumn{1}{l}{\hspace*{-2mm}$\uparrow$} &
\multicolumn{1}{l}{\hspace*{-2mm}$\uparrow$} \\
\multicolumn{1}{l}{\hspace*{-8mm}\tt ISTADA(1)} &
\multicolumn{1}{l}{\hspace*{-8mm}\tt ISTADA(2)} &
\multicolumn{1}{l}{\hspace*{-8mm}\tt ISTADA(3)} &
\multicolumn{1}{l}{\hspace*{-9mm}\tt ISTADA($n_g$)} &
\multicolumn{1}{l}{\hspace*{-7mm}\tt ISTADA($n_g+1$)}
\ectable{\label{2.3a}Contents of the arrays {\tt A}, {\tt ICNA} and
{\tt ISTADA}}
\vspace*{-4mm}

\itt{KNDOFG} is a rank-one allocatable array of dimension {\tt ng} and type
\integer, that is used to indicate which of the groups are to be
included in the objective function, which define equality constraints,
and which are to be ignored. If {\tt KNDOFG} is not
\associated, all groups will be included in the objective
function, and it will be assumed that there are no general constraints.
If {\tt KNDOFG} is \associated,
each of the first $n_g$ entries of {\tt KNDOFG} must
be set to $0$, $1$ or $2$ on initial entry.
If {\tt KNDOFG($i$)} has the value $1$, $i$ lies in
the set $\calG_O$ and the group
will be included in the objective function \req{objective}.
If, {\tt KNDOFG($i$)} has the value $2$, $i$ lies in the
complement set $\calG_C$, and the $i$-th group defines an equality constraint
of the form \req{constraints}. Finally, if
{\tt KNDOFG($i$)} has the value $0$, group $i$ will be ignored; this is
useful when, for example, many optimizations are required with
different subsets of constraints, or when a feasible point is sought
without reference to the objective.

\itt{ITYPEE} is a rank-one allocatable array of dimension {\tt nel} and type
\integer, that is used to indicate the types (See Section~\ref{ft})
of the nonlinear element functions. The $i$-th component of {\tt ITYPEE}
specifies the type of element $i$.

\itt{ISTEPA} is a rank-one allocatable array of dimension {\tt nel + 1} and type
\integer, whose $i$-th component gives the position in
{\tt EPVALU} (see below) of the first parameter for element function $i$.  In
addition, {\tt ISTEPA(nel+1)} is the position in
{\tt EPVALU} of the last  parameter for element function $n_e$ plus one. See
Table~\ref{3.2a} for an illustration.

\bctable{|c|c|c|c|c}
\cline{1-4}
parameters      & parameters & \hspace*{5mm} \ldots \hspace*{5mm} & parameters
   & {\tt EPVALU} \\
in $\bmp^e_1$ & in $\bmp^e_2$
&  & in $\bmp^e_{n_e}$ &              \\
\cline{1-4}
\multicolumn{1}{l}{\hspace*{-2mm}$\uparrow$} &
\multicolumn{1}{l}{\hspace*{-2mm}$\uparrow$} &
\multicolumn{1}{l}{\hspace*{-2mm}$\uparrow$} &
\multicolumn{1}{l}{\hspace*{-2mm}$\uparrow$} &
\multicolumn{1}{l}{\hspace*{-2mm}$\uparrow$} \\
\multicolumn{1}{l}{\hspace*{-8mm}\tt ISTEPA(1)} &
\multicolumn{1}{l}{\hspace*{-8mm}\tt ISTEPA(2)} &
\multicolumn{1}{l}{\hspace*{-8mm}\tt ISTEPA(3)} &
\multicolumn{1}{l}{\hspace*{-9mm}\tt ISTEPA($n_e$)} &
\multicolumn{1}{l}{\hspace*{-7mm}\tt ISTEPA($n_e+1$)}
\ectable{\label{3.2a}Contents of the arrays {\tt EPVALU} and {\tt ISTEPA}}
\vspace*{-4mm}

\itt{ITYPEG} is a rank-one allocatable array of dimension {\tt ng} and type
\integer, that is used to indicate the types (See Section~\ref{ft})
of the group functions. The $i$-th component of {\tt ITYPEG}
specifies the type of group $i$.

\itt{ISTGPA} is a rank-one allocatable array of dimension {\tt ng +1} and type
\integer, whose $i$-th component gives the position in
{\tt GPVALU} (see below) of the first parameter for group function $i$.  In
addition, {\tt ISTGPA(ng+1)} is the position in
{\tt GPVALU} of the last  parameter for element function $n_g$ plus one. See
Table~\ref{4.2a} for an illustration.

\bctable{|c|c|c|c|c}
\cline{1-4}
parameters      & parameters & \hspace*{5mm} \ldots \hspace*{5mm} & parameters
   & {\tt GPVALU} \\
in $\bmp^g_1$ & in $\bmp^g_2$
&  & in $\bmp^g_{n_g}$ &              \\
\cline{1-4}
\multicolumn{1}{l}{\hspace*{-2mm}$\uparrow$} &
\multicolumn{1}{l}{\hspace*{-2mm}$\uparrow$} &
\multicolumn{1}{l}{\hspace*{-2mm}$\uparrow$} &
\multicolumn{1}{l}{\hspace*{-2mm}$\uparrow$} &
\multicolumn{1}{l}{\hspace*{-2mm}$\uparrow$} \\
\multicolumn{1}{l}{\hspace*{-8mm}\tt ISTGPA(1)} &
\multicolumn{1}{l}{\hspace*{-8mm}\tt ISTGPA(2)} &
\multicolumn{1}{l}{\hspace*{-8mm}\tt ISTGPA(3)} &
\multicolumn{1}{l}{\hspace*{-9mm}\tt ISTGPA($n_g$)} &
\multicolumn{1}{l}{\hspace*{-7mm}\tt ISTGPA($n_g+1$)}
\ectable{\label{4.2a}Contents of the arrays {\tt GPVALU} and {\tt ISTGPA}}
\vspace*{-4mm}

\itt{A} is a rank-one array of dimension {\tt ISTADA(ng+1)-1} and type
\realdp, that holds the values of the nonzero components of the
gradients of the linear element functions,
$\bma_i$, $i = 1, \ldots, n_g$.
The values must appear in the same order as their indices appear in
{\tt ICNA}, i.e.,
the nonzero from element $i$, whose index is, say, {\tt ICNA($k$)}
will have value {\tt A($k$)}.
See Table~\ref{2.3a} and Section~\ref{galexample} for an example.

\itt{B} is a rank-one array of dimension {\tt ng} and type
\realdp, whose $i$-th entry holds the value of the constant
$b_i$ for each group.

\itt{BL} is a rank-one array of dimension {\tt n} and type
\realdp, whose $i$-th entry must be set
to the value of the lower bound $l_i$ on the $i$-th variable.
If the $i$-th variable has no lower bound, {\tt BL($i$)} should be set to
a large negative number.

\itt{BU} is a rank-one array of dimension {\tt n} and type
\realdp, whose $i$-th
entry must be set
to the value of the upper bound $u_i$ on the $i$-th variable.
If the $i$-th variable has no upper bound, {\tt BU($i$)} should be set to
a large positive number.

\itt{X} is a rank-one array of dimension {\tt n} and type
\realdp, that holds the current values of the
minimization variables, $\bmx$.

\itt{C} is a rank-one array of dimension {\tt ng} and type
\realdp, that holds the current estimates of the values of the
equality constraints
for the problem.
If {\tt KNDOFG($i$)} $<2$, {\tt C($i$)}
will not be set, but if {\tt KNDOFG($i$)} $= 2$, {\tt C($i$)} contains
the constraint value $c_i(\bmx)$
of the $i$-th constraint \req{constraints}.
{\tt C} need not be \associated\ if {\tt KNDOFG} is not.

\itt{Y} is a rank-one array of dimension {\tt ng} and type
\realdp, that holds the current estimates of the Lagrange multipliers,
$\bmy$, for the problem. If {\tt KNDOFG($i$)} $<2$, {\tt Y($i$)}
will not be set, but if {\tt KNDOFG($i$)} $= 2$, {\tt Y($i$)} contains
the multiplier estimate $y_i$
for the $i$-th constraint \req{constraints}.
{\tt Y} need not be \associated\ if {\tt KNDOFG} is not.

\itt{GSCALE} is a rank-one array of dimension {\tt ng} and type
\realdp, whose $i$-th entry holds
the value of $i$-th group weight $w^g_i$.

\itt{ESCALE} is a rank-one array of dimension {\tt ISTADG(ng+1)-1} and type
\realdp, whose entries hold the values of element weights
$w^e_{ij}$. The weights must occur in the same order as the indices of the
elements assigned to each group
in {\tt IELING}, with the weights for the elements
in group $i$ preceding those in group $i+1$, $i=1, \ldots, {\tt ng-1}$.
See Table~\ref{2.1a} and Section~\ref{galexample} for an example.

\itt{VSCALE} is a rank-one array of dimension {\tt n} and type
\realdp, that holds suitable positive scale
factors for the problem variables $\bmx$.
The {i}-th variable $x_i$ will implicitly be divided by
{\tt VSCALE($i$)} within \solver. The scale factors should ideally
be chosen so that the rescaled variables are of order one at the
solution to the minimization problem. If the user does not know
suitable scalings, each component of {\tt VSCALE} should be set to 1.0.
Good variable scalings can result in considerable savings in
computing times.

\itt{EPVALU} is a rank-one array of dimension {\tt ISTEPA(nel+1)-1}
and type \realdp, that holds
the values of the element parameters $\bmp^e_i$, $i = 1, \ldots,
n_e$. The indices for the parameters for element $i$ immediately precede
those for element $i+1$, and each element's parameters appear in a contiguous
list. See Table~\ref{3.2a} for an illustration.

\itt{GPVALU} is a rank-one array of dimension {\tt ISTGPA(ng+1)-1}
and type \realdp, that holds
the values of the group parameters $\bmp^g_i$, $i = 1, \ldots,
n_g$. The indices for the parameters for group $i$ immediately precede
those for group $i+1$, and each group's parameters appear in a contiguous
list. See Table~\ref{4.2a} for an illustration.

\itt{GXEQX} is a rank-one array of dimension {\tt ng} and type
default \logical, whose $i$-th entry must be
set \true\ if the $i$-th group function is the trivial function
$g(x)=x$ (see Section~\ref{ft}) and \false\ otherwise.

\itt{INTREP} is a rank-one array of dimension {\tt nel} and type
default \logical, whose $i$-th entry must be
set \true\ if the $i$-th nonlinear element function has a useful
transformation between elemental and internal variables
and \false\ otherwise (see Section~\ref{internal_and_ranges}).

\itt{VNAMES} is a rank-one array of dimension {\tt n} and type
default \character\ and length 10, whose $j$-th entry contains the
``name'' of the $j$-th variable.

\itt{GNAMES} is a rank-one array of dimension {\tt ng} and type
default \character\ and length 10, whose $i$-th entry contains the
``name'' of the $i$-th group.

\end{description}

%%%%%%%%%%% control type %%%%%%%%%

\subsubsection{The derived data type for holding control
 parameters}\label{typecontrol}
The derived data type
{\tt \fullpackagename\_control\_type}
is used to hold controlling data. Default values may be obtained by calling
{\tt \fullpackagename\_initialize}
(see Section~\ref{subinit}),
while components may also be changed by calling
{\tt \fullpackagename\_read\-\_spec}
(see Section~\ref{readspec}).
The components of
{\tt \fullpackagename\_control\_type}
are:

\begin{description}

\itt{error} is a scalar variable of type \integer, that holds the
stream number for error messages. Printing of error messages in
\solver\ and {\tt \fullpackagename\_terminate} is suppressed if
{\tt error} $\leq 0$.
The default is {\tt error = 6}.

\ittf{out} is a scalar variable of type \integer, that holds the
stream number for informational messages.
Printing of informational messages in
\solver\
is suppressed if {\tt out} $\leq$ 0.
The default is {\tt out} $=$ 6.

\itt{alive\_unit} is a scalar variable of type \integer.
If {\tt alive\_unit} $>$ 0, a temporary file named {\tt alive\_file} (see below)
will be created on stream number {\tt alive\_unit} on initial entry to
\solver, and execution of \solver\ will continue so
long as this file continues to exist. Thus, a user may terminate execution
simply by removing the temporary file from this unit.
If {\tt alive\_unit} $\leq$ 0, no temporary file will be created, and
execution cannot be terminated in this way.
The default is {\tt alive\_unit} $=$ 60.

\itt{alive\_file} is a scalar variable of type default \character\ and length
30, that gives the name of the temporary file whose removal from stream number
{\tt alive\_unit} terminates execution of \solver.
The default is {\tt alive\_unit} $=$ {\tt ALIVE.d}.

\itt{print\_level} is a scalar variable of type \integer, that is used
to control the amount of informational output which is required. No
informational output will occur if {tt print\_level} $\leq$ 0. If
{\tt print\_level} $= 1$, a single line of output will be produced for each
iteration of the process, while additionally if {\tt print\_level} $= 2$
a summary of the inner iteration will be given.
If {\tt print\_level} $\geq 3$, this output will be
increased to provide significant detail of each iteration.
The default is {\tt print\_level = 0}.

\itt{maxit} is a scalar variable of type \integer, that holds the
maximum number of iterations which will be allowed in \solver.
The default is {\tt maxit = 1000}.

\itt{start\_print} is a scalar variable of type \integer, that specifies
the first iteration for which printing will be permitted in \solver.
If {\tt start\_print} is negative, printing will be permitted from the outset.
The default is {\tt start\_print = -1}.

\itt{stop\_print} is a scalar variable of type \integer, that specifies
the last iteration for which printing will be permitted in  \solver.
If {\tt stop\_print} is negative, printing will be permitted once it has been
started by {\tt start\_print}.
The default is {\tt stop\_print = -1}.

\itt{print\_gap} is a scalar variable of type \integer.
Once printing has been started, output will occur once every
{\tt print\_gap} iterations. If {\tt print\_gap} is no larger than 1,
printing will be permitted on every iteration.
The default is {\tt print\_gap = 1}.

\itt{linear\_solver} is a scalar variable of type \integer, that
is used to specify the method used to solve the linear systems of equations
which arise at each iteration of the minimization algorithm.
 Possible values are:
\begin{description}
\itt{1$.$}
    The conjugate gradient method will be used without preconditioning.
\itt{2$.$}
    A preconditioned conjugate gradient
    method will be used with a diagonal
    preconditioner.
\itt{3$.$}
    A preconditioned conjugate gradient
    method will be used with a user supplied
    preconditioner. Control will be passed back to the user to
    construct the product of the preconditioner with a given
    vector prior to a reentry to \solver\
    (see Section~\ref{reverse}).
\itt{4$.$}
    A preconditioned conjugate gradient
    method will be used with an expanding band
    incomplete Cholesky preconditioner.
    This option is only available if the user has provided one of the
    external packages {\tt MA27} or {\tt HSL\_MA57},
    (see Section~\ref{galexternal}).
\itt{5$.$}
    A preconditioned conjugate gradient
    method will be used with Munksgaard's
    preconditioner.
    This option is only available if the user has provided the
    external package {\tt MA61}
    (see Section~\ref{galexternal}).
\itt{6$.$}
    A preconditioned conjugate gradient
    method will be used with a modified Cholesky
    preconditioner due to Gill, Murray, Poncel\'{e}on and Saunders,
    in which an indefinite factorization is altered to give a positive 
    definite one.
    This option is only available if the user has provided one of the
    external packages {\tt MA27} or {\tt HSL\_MA57},
    (see Section~\ref{galexternal}).
\itt{7$.$}
    A preconditioned conjugate gradient
    method will be used with a modified Cholesky
    preconditioner due to Schnabel and Eskow, 
    in which small or negative diagonals are
    made sensibly positive during the factorization. 
    This option is only available if the user has provided one of the
    external packages {\tt MA27} or {\tt HSL\_MA57},
    (see Section~\ref{galexternal}).
\itt{8$.$}
    A preconditioned conjugate gradient
    method will be used with a band
    preconditioner. The semi-bandwidth
    is given by the variable {\tt semibandwidth} (see below).
\itt{9$.$}
    A preconditioned conjugate gradient
    method will be used with an incomplete Cholesky factorization
    preconditioner due to Lin and Mor\'{e}.
    This option is only
    available if the user has provided the external package {\tt ICFS},
    (see Section~\ref{galexternal}).
\itt{11$.$}
    A multifrontal factorization method will be used.
    This option is only available if the user has provided one of the
    external packages {\tt MA27} or {\tt HSL\_MA57},
    (see Section~\ref{galexternal}).
\itt{12$.$}
    A modified Cholesky factorization method will be used.
    This option is only available if the user has provided one of the
    external packages {\tt MA27} or {\tt HSL\_MA57},
    (see Section~\ref{galexternal}).
\end{description}
Any other value of {\tt linear\_solver} will be regarded as {\tt 1}.
The default is {\tt linear\_solver = 8}.

\itt{icfact} is a scalar variable of type \integer, that specifies
the number of extra vectors of length {\tt prob\%n} used by the Lin and
Mor\'{e} preconditioner ({\tt linear\_solver = 4}) if requested.
Usually, the larger the number, the
better the preconditioner, but the more space and effort required to
use it. Any negative value will be regarded as {\tt 0}.
The default is {\tt icfact = 5}.

\itt{semibandwidth} is a scalar variable of type \integer, that holds
the semi-bandwidth used if the band preconditioner ({\tt linear\_solver = 8})
is selected. Any negative value will be regarded as {\tt 0}.
The default is {\tt semibandwidth = 5}.

\itt{max\_sc} is a scalar variable of type \integer, that specifies
the maximum number of variables that are allowed to hit their bounds
during the calculation of the step before a refactorization of the
preconditioner is triggered. A dense Schur-complement matrix of order
at most {\tt max\_sc} may be generated in lieu of the refactorization, so
some compromise between many refactorizations and the storage of a potentially
large, dense Schur complement must be made.
Any non-positive value will be regarded as {\tt 1}.
The default is {\tt max\_sc = 75}.

\itt{io\_buffer} is a scalar variable of type \integer, that holds
the unit number of an input/output buffer for writing temporary files
during array-size re-allocations, if needed.
The default is {\tt io\_buffer = 75}.

\itt{more\_toraldo} is a scalar variable of type \integer, that
specifies the number of Mor\'{e}-Toraldo projected searches
that are to be used to improve upon the Cauchy point when finding the
step. Any non-positive value results in a standard add-one-at-a-time
conjugate-gradient search.
The default is {\tt more\_toraldo = 0}.

\itt{non\_monotone} is a scalar variable of type \integer, that
specifies the history-length for non-monotone descent strategy.
Any non-positive value results in standard monotone descent, for which
merit function improvement occurs at each iteration. There are often
definite advantages in using a non-monotone strategy with a modest history,
since the occasional local increase in the merit function may
enable the algorithm to move across (gentle) ``ripples'' in
the merit function surface.
However, we do not usually recommend large values of {\tt non\_monotone}.
The default is {\tt non\_monotone = 1}.

\itt{first\_derivatives} is a scalar variable of type \integer,
that specifies what sort of first derivative
approximation to use. If the user is able to provide
analytical first derivatives for each nonlinear
element function, {\tt first\_derivatives} must be set to
to be non-positive. If analytical first derivatives are unavailable,
they may be estimated by forward differences by setting
{\tt first\_derivatives = 1}, or, more accurately but at additional expense,
by central differences by setting {\tt first\_derivatives} $\geq$ {\tt 2}.
The default is {\tt first\_derivatives = 0}.
{\bf N.B.} This value will be ignored whenever the user provides the
optional argument {\tt ELDERS} to \solver.

\itt{second\_derivatives} is a scalar variable of type \integer,
that specifies what sort of second derivative
approximation to use. If the user is able to
provide analytical second derivatives for each nonlinear element function,
{\tt second\_derivatives} must be set to be non-positive. If the
user is unable to provide second derivatives, these
derivatives will be approximated using one of four
secant approximation formulae. If {\tt second\_derivatives} is set to
1, the BFGS formula is used; if it is set to 2, the DFP
formula is used; if it is set to 3, the PSB
formula is used; and if it is set to 4 or larger, the symmetric
rank-one formula is used. The user is strongly advised to
use exact second derivatives
if at all possible as this often
significantly improves the convergence of the method.
The default is {\tt second\_derivatives = 0}.
{\bf N.B.} This value will be ignored whenever the user provides the
optional argument {\tt ELDERS} to \solver.

\itt{stopg} is a scalar variable of type \realdp,
that is used to specify the maximum permitted (infinity)
norm of the projected gradient of the Lagrangian function
(see Section~\ref{galmethod}) at the estimate of the solution sought.
The default is {\tt stopg =} $10^{-5}$.

\itt{stopc} is a scalar variable of type \realdp, that is used to
specify the maximum permitted violation (measured in infinity
norm) of the constraints at the estimate of the solution sought.
The default is {\tt stopc =} $10^{-5}$.

\itt{min\_aug} is a scalar variable of type \realdp, that is used
to specify the smallest permitted value of the merit function
\req{objectiveb}. Any smaller value will result in the minimization
being terminated. The default is {\tt min\_aug = - HUGE(1.0)/8.0}.

\itt{acccg} is a scalar variable of type \realdp, that is used to
specify the least relative reduction in (two-) norm of the residual
of the reduced gradient of the model function that is required
from the conjugate gradient iteration.
The default is {\tt acccg = 0.01}.

\itt{initial\_radius} is a scalar variable of type \realdp, that holds
the required initial value of the trust-region radius. If
{\tt initial\_radius} $\leq 0$, the radius will be chosen automatically
by \solver\
The default is {\tt initial\_radius = - 1.0}.

\itt{maximum\_radius} is a scalar variable of type \realdp, that holds
the maximum permitted value of the trust-region radius. Any radius that
exceeds the larger of {\tt maximum\_radius} and one
during the calculation will be reset to
{\tt MAX(maximum\_radius,1.0)}.
The default is {\tt maximum\_radius =} $10^{20}$.

\itt{eta\_successful\rm ,} {\tt eta\_very\_successful}
and {\tt eta\_extremely\_successful} are scalar variables of type default
\realdp, that control the acceptance and rejection of the trial step
and the updates to the trust-region radius.
At every iteration, the ratio of the actual reduction in the merit function
following the trial step to that predicted by the model is computed.
The step is accepted whenever this ratio exceeds {\tt eta\_successful};
otherwise the trust-region radius will be reduced.
If, in addition, the ratio exceeds {\tt eta\_very\_successful}, the trust-region
radius may be increased. If a structured trust-region is being used
(see {\tt structured\_tr} below), the radius for an individual element
will only be increased if additionally the ratio of the actual to
predicted decrease for this element exceeds {\tt eta\_extremely\_successful}.
The defaults are
{\tt eta\_successful = 0.01},
{\tt eta\_very\_successful = 0.9} and
{\tt eta\_extremely\_successful = 0.95}.

\itt{gamma\_smallest\rm ,} {\tt gamma\_decrease} and
{\tt gamma\_increase} are scalar variables of type \realdp, that
control the maximum amounts by which the trust-region radius can
contract or expand during an iteration. The radius will be decreased by
powers of {\tt gamma\_decrease} until it is smaller than the larger
of an internally calculated prediction of what should be a good value
and {\tt gamma\_smallest}. It can be increased by at most a factor
{\tt gamma\_increase}.
The defaults are
{\tt gamma\_smallest = 0.0625},
{\tt gamma\_decrease = 0.25} and
{\tt gamma\_increase = 2.0}.

\itt{mu\_meaningful\_model} and
{\tt mu\_meaningful\_group} are scalar variables of type \realdp, that
hold tolerances that determine whether a group (and its model)
is ``meaningful'' when a structured trust-region is being used
(see {\tt structured\_tr} below). A model of an individual group
is meaningful if the change in its value
that results from the trial step is more than a factor
{\tt mu\_meaningful\_model} of the overall model decrease. The
same is true of the group itself if if the change in its value
that results from the trial step is more than a factor
{\tt mu\_meaningful\_group} of the overall model decrease.
The radius update strategies are more liberal for non-meaningful
groups (see Conn, Gould, Toint, 2000, Section 10.2.2).
These values are for experts only.
The defaults are
{\tt mu\_meaningful\_model = 0.01} and
{\tt mu\_meaningful\_group = 0.1}.

\itt{initial\_mu} is a scalar variable of type \realdp, that specifies
the initial value of the penalty parameter, $\mu$. Any value smaller than
$u$ will be replaced by $u$,
where $u$ is {\tt EPSILON(1.0)} ({\tt EPSILON(1.0D0)} in
{\tt \fullpackagename\_dou\-ble}).
The default is {\tt initial\_mu = 0.1}.

\itt{mu\_tol} is a scalar variable of type \realdp, that holds a
threshold value of the penalty parameter, above which no attempt will be
made to update Lagrange multiplier estimates.
The default is {\tt mu\_tol = 0.1}.

\itt{firstg} is a scalar variable of type \realdp, that is used to
specify the maximum permitted (infinity)
norm of the projected gradient of the Lagrangian function
(see Section~\ref{galmethod}) at the end of the first major iteration.
The default is {\tt firstg = 0.1}.

\itt{firstc} is a scalar variable of type \realdp, that is used to
specify the maximum hoped-for violation (measured in infinity
norm) of the constraints at the end of the first major iteration.
No attempt will be made to update Lagrange multiplier estimates until the
violation falls below this threshold.
The default is {\tt firstc = 0.1}.

\itt{cpu\_time\_limit} is a scalar variable of type \realdp,
that is used to specify the maximum permitted CPU time. Any negative
value indicates no limit will be imposed. The default is
{\tt cpu\_time\_limit = - 1.0}.

\itt{quadratic\_problem} is a scalar variable of type default \logical, that
should be set \true\ if the objective function is quadratic (or linear)
and any constraints linear, and \false\ otherwise. {\tt \packagename} can
take some advantage of quadratic problems, although the \galahad\ packages
{\tt \libraryname\_QPB} or {\tt \libraryname\_QPA} are to be preferred
in this case.
The default is {\tt quadratic\_problem = \false}.

\itt{two\_norm\_tr} is a scalar variable of type default \logical, that should
be set \true\ if a two-norm (hyperspherical) trust region is
required, and \false\ if an infinity-norm (box)
trust region is to be used.
The default is {\tt two\_norm\_tr = \false}.

\itt{exact\_gcp} is a scalar variable of type default \logical, that should
be set \true\ if the exact generalized Cauchy point,
the first estimate of the minimizer of the quadratic model within
the feasible box, is required, and \false\ if an approximation suffices.
The default is {\tt exact\_gcp = \true}.

\itt{accurate\_bqp} is a scalar variable of type default \logical, that should
be set \true\ if an accurate minimizer of the
quadratic model within the feasible box is required, and
\false\ if an approximation suffices. The accurate minimizer often
requires considerably more work, but occasionally this reduces the overall
number of iterations.
The default is {\tt accurate\_bqp = \false}.

\itt{structured\_tr} is a scalar variable of type default \logical, that should
be set \true\ if the shape of the trust region should be adjusted to
account for the partial separability of the problem,
and \false\ otherwise.
The default is {\tt structured\_tr = \false}.

\itt{print\_max} is a scalar variable of type default \logical, that should
be set \true\ if the {\em printed} values of the objective function and its
derivatives should be multiplied by minus one (as might be the case if
we aim to maximize $f(\bmx)$ by minimizing $-f(\bmx)$), and \false\ otherwise.
The default is {\tt print\_max = \false}.

\itt{full\_solution} is a scalar variable of type default \logical, that should
be set \true\ if all components of the solution $\bmx$ and constraints
$\bmc(\bmx)$ should be printed on termination when {\tt print\_level > 0},
and \false\ if only the first and last (representative) few are required.
The default is {\tt full\_solution = \true}.

\itt{space\_critical} is a scalar variable of type default \logical,
that must be set \true\ if space is critical when allocating arrays
and  \false\ otherwise. The package may run faster if
{\tt space\_critical} is \false\ but at the possible expense of a larger
storage requirement. The default is {\tt space\_critical = .FALSE.}.

\itt{deallocate\_error\_fatal} is a scalar variable of type default \logical,
that must be set \true\ if the user wishes to terminate execution if
a deallocation  fails, and \false\ if an attempt to continue
will be made. The default is {\tt deallocate\_error\_fatal = .FALSE.}.

\itt{SILS\_cntl} is a scalar variable of type {\tt SILS\_control}, that holds
control parameters for the \galahad\ sparse matrix factorization package
{\tt SILS}. See the documentation for the package {\tt SILS} for details of the
derived type {\tt SILS\_control}, and of the default values given to
its components.

\end{description}

%%%%%%%%%%% data type %%%%%%%%%

\subsubsection{The derived data type for holding problem data}\label{typedata}
The derived data type
{\tt \fullpackagename\_data\_type}
is used to hold all the data for a particular problem between calls of
{\tt \fullpackagename} procedures.
This data should be preserved, untouched, from the initial call to
{\tt \fullpackagename\_initialize}
to the final call to
{\tt \fullpackagename\_terminate}.

%%%%%%%%%%% info type %%%%%%%%%

\subsubsection{The derived data type for holding informational
 parameters}\label{typeinfo}
The derived data type
{\tt \fullpackagename\_info\_type}
is used to hold parameters that give information about the progress and needs
of the algorithm. The components of
{\tt \fullpackagename\_info\_type}
are:

\begin{description}
\itt{status} is a scalar variable of type \integer, that gives the
exit status of the algorithm. See Sections~\ref{reverse} and
\ref{galerrors} for details. A value {\tt status = 0} indicates
successful termination.

\itt{alloc\_status} is a scalar variable of type \integer, that gives
the status of the last attempted array allocation or de-allocation.

\itt{iter} is a scalar variable of type \integer, that gives
the number of iterations that have been performed since the start of the
minimization. This is one fewer than the number of
problem function evaluations that have been made.

\itt{itercg} is a scalar variable of type \integer, that gives
the number of conjugate-gradient iterations that have
been performed since the start of the minimization.

\itt{itcgmx} is a scalar variable of type \integer, that gives
the maximum number of conjugate-gradient iterations that will be allowed
at each iteration of the inner minimization.

\itt{ncalcf} is a scalar variable of type \integer, that gives
the number of element functions that must be re-evaluated when the
reverse-communication mode is used to evaluate element values.

\itt{ncalcg} is a scalar variable of type \integer, that gives
the number of group functions that must be re-evaluated when the
reverse-communication mode is used to evaluate group values.

\itt{nvar} is a scalar variable of type \integer, that gives
the current number of free variables, that is the number of variables
which do not lie on either simple bounds.

\itt{ngeval} is a scalar variable of type \integer, that gives
the number of problem function gradient evaluations that have
been made since the start of the minimization.

\itt{iskip} is a scalar variable of type \integer, that gives
the number of times that an approximation to the
second derivatives of an element function has been rejected.

\itt{ifixed} is a scalar variable of type \integer, that gives
the index of the variable that most recently
encountered one of its bounds in the minimization process.

\itt{nsemib} is a scalar variable of type \integer, that gives
the bandwidth used if the band
or expanding band preconditioner is selected.
({\tt control\%linear\_solver = 4} or {\tt 8}).

\itt{aug} is a scalar variable of type \realdp, that gives
the current value of the augmented Lagrangian merit function.

\itt{obj} is a scalar variable of type \realdp, that gives
the current value of the objective function.

\itt{pjgnrm} is a scalar variable of type \realdp, that gives
the (infinity) norm of the current projected gradient of the Lagrangian
function (see Section~\ref{galmethod}).

\itt{cnorm} is a scalar variable of type \realdp, that gives
the (infinity) norm of the current violation of the constraints.

\itt{ratio} is a scalar variable of type \realdp, that gives
the ratio of the actual reduction that has been made in the
merit function value during the current iteration to that
predicted by the model function.  A value close to one is to be expected
as the algorithm converges.

\itt{mu} is a scalar variable of type \realdp, that gives
the current value of the penalty parameter.

\itt{radius} is a scalar variable of type \realdp, that gives
the current value of the radius of the trust-region.

\itt{ciccg} is a scalar variable of type \realdp, that gives
the current value of the pivot tolerance that is
used when Munksgaard's incomplete Cholesky factorization
({\tt control\%linear\_sol\-ver = 5}) is chosen for preconditioning.

\itt{newsol} is a scalar variable of type default \logical, that will be
\true\ if a major iteration has just been completed, and \false\ otherwise.

\itt{bad\_alloc} is a scalar variable of type
default \character\ and length 80, that gives
the name of the last array for which an array allocation/de-allocation
was unsuccessful.

\itt{SCU\_info} is a scalar variable of type {\tt SCU\_info\_type}, that holds
informational parameters concerning the
the \galahad\ bordered matrix factorization package {\tt SCU}.
See the documentation for the package {\tt SCU} for details of the
derived type {\tt SCU\_info\_type}.

\itt{SILS\_infoa} is a scalar variable of type {\tt SILS\_ainfo}, that holds
informational parameters concerning the analysis subroutine contained in
the \galahad\ sparse matrix factorization package {\tt SILS}.
See the documentation for the package {\tt SILS} for details of the
derived type {\tt SILS\_ainfo}.

\itt{SILS\_infof} is a scalar variable of type {\tt SILS\_finfo}, that holds
informational parameters concerning the factorization subroutine contained in
the \galahad\ sparse matrix factorization package {\tt SILS}.
See the documentation for the package {\tt SILS} for details of the
derived type {\tt SILS\_finfo}.

\end{description}

%%%%%%%%%%%%%%%%%%%%%% argument lists %%%%%%%%%%%%%%%%%%%%%%

\galarguments
There are three principal procedures for user calls
(see Section~\ref{galfeatures} for further features):

\begin{enumerate}
\item The subroutine
      {\tt \fullpackagename\_initialize}
      is used to set default values, and initialize private data,
      before solving the specified optimization problem.
\item The subroutine
      \solver\
      is called to solve the optimization problem.
\item The subroutine
      {\tt \fullpackagename\_terminate}
      is provided to allow the user to automatically deallocate array
       components of the private data, allocated by
       \solver,
       at the end of the solution process.
\end{enumerate}
We use square brackets {\tt [ ]} to indicate \optional\ arguments.

%%%%%% initialization subroutine %%%%

\subsubsection{The initialization subroutine}\label{subinit}
 Default values are provided as follows:
\hskip0.5in
\def\baselinestretch{0.8} {\tt \begin{verbatim}
     CALL LANCELOT_initialize( data, control )
\end{verbatim}}
\def\baselinestretch{1.0}

\begin{description}
\itt{data} is a scalar \intentinout argument of type
{\tt \fullpackagename\_data\_type}
(see Section~\ref{typedata}). It is used to hold data about the problem being
solved.

\itt{control} is a scalar \intentout argument of type
{\tt \fullpackagename\_control\_type}
(see Section~\ref{typecontrol}).
On exit, {\tt control} contains default values for the components as
described in Section~\ref{typecontrol}.
These values should only be changed after calling
{\tt \fullpackagename\_initialize}.

\end{description}

%%%%%%%%% main solution subroutine %%%%

\subsubsection{The minimization subroutine\label{min}}
The minimization algorithm is called as follows:
\hskip0.5in
\def\baselinestretch{0.8} {\tt \begin{verbatim}
     CALL LANCELOT_solve( prob, RANGE, GVALS, FT, XT, FUVALS, lfuval, ICALCF, ICALCG, &
                          IVAR, Q, DGRAD, control, inform, data[, ELDERS] [, ELFUN]   &
                          [, ELFUN_flexible] [, GROUP] )
\end{verbatim}}
\def\baselinestretch{1.0}

\begin{description}

\ittf{prob} is a scalar \intentinout argument of type
{\tt \fullpackagename\_problem\_type}
(see Section~\ref{typeprob}).
It is used to hold data about the problem being solved.
Components
{\tt n},
{\tt ng},
{\tt nel},
{\tt IELING},
{\tt ISTADG},
{\tt IELVAR},
{\tt ISTAEV},
{\tt ICNA},
{\tt ISTADA},
{\tt A},
{\tt B},
{\tt BL},
{\tt BU},
{\tt GSCALE},
{\tt ESCALE},
{\tt VSCALE},
{\tt GXEQX},
{\tt INTREP},
{\tt VNAMES}, and
{\tt GNAMES}
must all be set on entry as described in Section~\ref{typeprob}, and
will thereafter be unaltered.

The component {\tt INTVAR} must also be set as
described in Section~\ref{typeprob}, but will subsequently be reset by \solver\
so that its $i$-th value
gives the position in the array {\tt FUVALS} (see below)
of the first component of
the gradient of the $i$-th nonlinear element function, {\em with respect
to its internal variables} (see {\tt FUVALS}).
The component {\tt X} must also be set, but will be altered by \solver.
The component {\tt ISTADH} need not be set on initial entry, but will
be set by \solver\ as described in Section~\ref{typeprob}.

If the dummy argument {\tt ELFUN} (see below) is present,
the components {\tt ISTEPA} and {\tt EPVALU} must also be
set on entry as described in Section~\ref{typeprob}---they will
not subsequently be altered---but otherwise they need not be
associated or assigned.
Likewise if the dummy argument {\tt GROUP} (see below) is present,
the components {\tt ISTGPA}, {\tt GPVALU} and {\tt ITYPEG} must also be
set on entry as described in Section~\ref{typeprob}---they will
not subsequently be altered---but otherwise they need not be
associated or assigned.

If the problem involves general constraints \req{constraints},
the components {\tt KNDOFG}, {\tt Y} and {\tt C} must be associated, and
the first two assigned values---{\tt KNDOFG} will not subsequently be
altered, while {\tt C} will be set by \solver.
If the problem does not involve general
constraints, {\tt KNDOFG}, {\tt Y} and {\tt C}
need not be associated.

\itt{RANGE} is a user-supplied subroutine whose purpose is to
define the linear transformation of variables for those nonlinear
elements which have different elemental and internal variables.
See Section~\ref{internal_and_ranges} for details.
{\tt RANGE}  must be declared {\tt EXTERNAL} in the calling program.

\itt{GVALS} is a rank-two \intentout array argument of
shape {\tt (prob\%ng, 3)} and type \realdp\,
that is used to store function and derivative information for the
group functions. The user may be asked to provide values for these functions
and/or derivatives, evaluated at
the argument {\tt FT} (see {\tt FT})
when control is returned to the calling
program with a negative value of the variable {\tt inform\%status}.
This information needs to be stored by the user
in specified locations within {\tt GVALS}.
Details of the required information are given in Section~\ref{reverse}.

\itt{FT} is a rank-one \intentinout array argument of
dimension {\tt prob\%ng} and type \realdp\, that is set within \solver\
to a trial value of the argument of the $i$-th group function
at which the user may be required to evaluate the values and/or derivatives
of that function. Precisely what group function information
is required at {\tt FT} is under the control of the
variable {\tt inform\%status} and details are given in Section~\ref{reverse}.

\itt{XT} is a rank-one \intentinout array argument of
dimension {\tt prob\%n} and type \realdp\, that is set within \solver\
to a trial value of the variables $\bmx$ at which the
user may be required to evaluate the values and/or derivatives
of the nonlinear elements functions. Precisely what element function
information is required at {\tt XT} is under the control of the
variable {\tt inform\%status} and details are given in Section~\ref{reverse}.

\itt{FUVALS} is a rank-one \intentinout array argument of
dimension {\tt lfuval} and type \realdp\, that is used to store
function and derivative information for the nonlinear element functions.
The user is asked to provide values for these functions and/or derivatives,
evaluated at the argument {\tt XT} (see {\tt XT}), at specified
locations within {\tt FUVALS}, when control is returned to the calling
program with a negative value of the variable {\tt inform\%status}.
Alternatively, the user may have provided a suitable subroutine {\tt ELFUN}
to compute the required function or derivative values (see below). Details
of the required information are given in Sections~\ref{efv} and \ref{reverse}.
The layout of {\tt FUVALS} is indicated in Table~\ref{2.4a}.

{\scriptsize
\bctable{|c|c|c|c|c|l}
\multicolumn{1}{c}{$\longleftarrow$~~~$n_e$~~~$\longrightarrow$} &
\multicolumn{1}{c}{} &
\multicolumn{1}{c}{} &
\multicolumn{1}{c}{$\longleftarrow$~~~$n$~~~$\longrightarrow$} &
\multicolumn{1}{c}{$\longleftarrow$~~~$n$~~~$\longrightarrow$} &
\\
\cline{1-5}
element & \hspace*{1mm} element \hspace*{1mm}   &
\hspace*{1mm} element \hspace*{1mm}  & objective & Hessian  \\
values  & gradients & Hessians & gradient  & diagonal \\
\cline{1-5}
\multicolumn{1}{l}{} &
\multicolumn{1}{l}{\hspace*{-2mm}$\uparrow$} &
\multicolumn{1}{l}{\hspace*{-2mm}$\uparrow$} &
\multicolumn{1}{l}{\hspace*{-2mm}$\uparrow$} &
\multicolumn{1}{l}{} &
\multicolumn{1}{l}{\hspace*{-2mm}$\uparrow$} \\
\multicolumn{1}{l}{} &
\multicolumn{1}{l}{\hspace*{-8mm}\tt INTVAR(1)} &
\multicolumn{1}{l}{\hspace*{-8mm}\tt ISTADH(1)} &
\multicolumn{1}{l}{\hspace*{-8mm}\tt ISTADH($n_e+1$)} &
\multicolumn{1}{l}{} &
{\hspace*{-6mm}\tt lfuval}
\ectable{\label{2.4a}Partitioning of the workspace array {\tt FUVALS}}
}
\vspace*{-4mm}

The first segment of {\tt FUVALS} contains the values of the nonlinear
element functions; the next two segments contain their gradients and
Hessian matrices, taken with respect to their internal variables, as
indicated in Section~\ref{reverse}.  The remaining two used segments
contain the gradient of the objective function and the diagonal elements
of the second derivative approximation, respectively.  At the solution,
the components of the gradient of the augmented Lagrangian function
corresponding to variables which lie on one of their bounds are of
particular interest in many applications areas.  In particular they are
often called shadow prices, and are used to assess the sensitivity of
the solution to variations in the bounds on the variables.

\itt{lfuval} is a scalar \intentin\ argument of type \integer,
that must be set to the actual length of {\tt FUVALS} in the calling program.
The required length may be calculated using the following code segment:
\begin{verbatim}
   lfuval = prob%nel + 2 * prob%n
   DO i = 1, prob%nel
     lfuval = lfuval + ( prob%INTVAR( i ) * ( prob%INTVAR( i ) + 3 ) ) / 2
   END DO
\end{verbatim}

\itt{ICALCF} is a rank-one \intentinout array argument of
dimension {\tt prob\%nel} and type \integer.
If the user has chosen not to perform internal element evaluations,
and if the value of {\tt inform\%status} on return from \solver\
indicates that further element functions values or their derivatives
are required prior to a re-entry,
the first {\tt inform\%ncalcf} components of {\tt ICALCF} give the
indices of the group functions which need to be
recalculated at {\tt XT}.
Precisely what group function information
is required is under the control of the
variable {\tt inform\%status} and details are given in Section~\ref{reverse}.

\itt{ICALCG} is a rank-one \intentinout array argument of
dimension {\tt prob\%ng} and type \integer.
If the user has chosen not to perform internal group evaluations,
and if the value of {\tt inform\%status} on return from \solver\
indicates that further group functions values or their derivatives
are required prior to a re-entry,
the first {\tt inform\%ncalcg} components of {\tt ICALCG} give the
indices of the group functions which need to be
recalculated at {\tt FT}.
Precisely what group function information
is required is under the control of the
variable {\tt inform\%status} and details are given in Section~\ref{reverse}.

\itt{IVAR} is a rank-one \intentinout array argument of
dimension {\tt prob\%n} and type \integer\, that is
required when the user is providing a special preconditioner
for the conjugate gradient inner iteration (see Section~\ref{reverse}).

\itt{Q} and {\tt DGRAD} are rank-one \intentinout array arguments of
dimension {\tt prob\%n} and type \realdp\, that are
required when the user is providing a special preconditioner
for the conjugate gradient inner iteration
(see Section~\ref{reverse}).

\itt{control} is a scalar \intentinout argument of type
{\tt \fullpackagename\_control\_type}
(see Section~\ref{typecontrol}). On exit, {\tt control} contains default
values for the components as described in Section~\ref{typecontrol}.
These values should only be changed after calling
{\tt \fullpackagename\_initialize}.

\itt{inform} is a scalar \intentinout argument of type
{\tt \fullpackagename\_type}
(see Section~\ref{typeinfo}). The component {\tt status} must be set to
{\tt 0} on initial entry, and a successful call to
\solver\ is indicated when the  component {\tt status} has the value {\tt 0}.
For other return values of {\tt status}, see Sections~\ref{reverse} and
\ref{galerrors}.

\itt{data} is a scalar \intentinout argument of type
{\tt \fullpackagename\_data\_type}
(see Section~\ref{typedata}). It is used to hold data about the problem being
solved. It must not have been altered {\bf by the user} since the last call to
{\tt \fullpackagename\_initialize}.


\itt{ELDERS} is an \optional\ rank-two \intentinout array argument of
shape {\tt (2, prob\%nel)} and type \integer\,
that may be used to specify what kind of first- and second-derivative
approximations will be required for each nonlinear element function.
If {\tt ELDERS} is not present, the first- and second-derivative
requirements for every element will be as specified by
the variables {\tt control\%first\_derivatives} and
{\tt control\%second\_derivatives} (see Section~\ref{typecontrol})
respectively. For finer control,
if {\tt ELDERS} is present, and the user is able to provide
analytical first derivatives for the $i$-th nonlinear
element function, {\tt ELDERS(1,i)}
must be set to to be non-positive.
If analytical first derivatives are unavailable,
they may be estimated by forward differences by setting
{\tt ELDERS(1,i) = 1} or, more accurately but at additional expense,
by central differences by setting
{\tt ELDERS(1,i) $\geq$ 2}.
Similarly, if {\tt ELDERS} is present, and the user is able to provide
analytical second derivatives for the $i$-th nonlinear
element function, {\tt ELDERS(2,i)}
must be set to to be non-positive.
If the user is unable to provide second derivatives, these
derivatives will be approximated using one of four
secant approximation formulae. If {\tt ELDERS(2,i)} is set to
1, the BFGS formula is used; if it is set to 2, the DFP
formula is used; if it is set to 3, the PSB
formula is used; and if it is set to 4 or larger, the symmetric
rank-one formula is used. The user is strongly advised to
use analytic first and second derivatives
if at all possible as this often
significantly improves the convergence of the method.

\itt{ELFUN} and {\tt ELFUN\_flexible} are \optional\
user-supplied subroutines whose purpose is to
evaluate the values and derivatives of the nonlinear element functions.
See Section~\ref{pfe} for background information and Sections~\ref{efv}
and \ref{efvf} for details.
Only one of {\tt ELFUN} and {\tt ELFUN\_flexible} may be present at once,
and, if present, must be declared {\tt EXTERNAL} in the calling program.
Which of the two arguments is permitted depends on whether {\tt ELDERS}
(see above) is present. {\tt ELFUN\_flexible} is only permitted when
{\tt ELDERS} is present, since {\tt ELFUN\_flexible} allows the user
to provide element-specific levels of derivative information. In the
absence of {\tt ELDERS},  {\tt ELFUN} should be used instead of
{\tt ELFUN\_flexible}.
If both {\tt ELFUN} and {\tt ELFUN\_flexible}
are absent, \solver\ will use reverse communication to
obtain element function values and derivatives.

\itt{GROUP} is an \optional\
user-supplied subroutine whose purpose is to
evaluate the values and derivatives of the group functions.
See Section~\ref{pfe} for background information and Section~\ref{gfv}
for details.
If {\tt GROUP} is present,
it must be declared {\tt EXTERNAL} in the calling program.
If {\tt GROUP} is absent, \solver\ will use reverse communication to
obtain group function values and derivatives. {\tt GROUP} need not be
present if all components of {\tt prob\%GXEQX} are \true.

\end{description}

%%%%%%% termination subroutine %%%%

\subsubsection{The  termination subroutine}
All previously allocated arrays are deallocated as follows:
\hskip0.5in
\def\baselinestretch{0.8} {\tt \begin{verbatim}
     CALL LANCELOT_terminate( data, control, inform )
\end{verbatim}}
\def\baselinestretch{1.0}

\begin{description}
\itt{data} is a scalar \intentinout argument of type
{\tt \fullpackagename\_data\_type}
exactly as for
\solver\
that must not have been altered {\bf by the user} since the last call to
{\tt \fullpackagename\_initialize}.

\itt{control} is a scalar \intentin argument of type
{\tt \fullpackagename\_control\_type}
exactly as for
\solver.

\itt{inform} is a scalar \intentinout argument of type
{\tt \fullpackagename\_type}
exactly as for
\solver.
Only the components {\tt status}, {\tt alloc\_status} and
{\tt bad\_alloc} might have been altered on exit, and a
successful call to
{\tt \fullpackagename\_terminate}
is indicated when this  component {\tt status} has the value 0.
For other return values of {\tt status}, see Section~\ref{galerrors}.

\end{description}

%%%%%%%%%%%%%%%%%%%% Function and derivative values %%%%%%%%%%%%%%%%%%%%%

\subsection{Function and derivative values\label{fdv}}

%%%%%%%%%%%%%%%%%%%%%% Problem function evaluation %%%%%%%%%%%%%%%%%%%%%%

\subsubsection{Evaluating problem functions\label{pfe}}

As we saw in Section~\ref{galbasics},
both the objective function \req{objective} and constraints \req{constraints}
are composed from element functions $e_j(\bmx^e_j, \bmp^e_j)$ and
group functions $g_i( \alpha^{ }_i, \bmp^g_i )$, all other defining data
being linear (and thus described accurately on initial input
to \solver\ by components of {\tt prob})---here we refer to the {\em scalar}
\disp{\alpha_i \eqdef \sum_{j \in \calE_i} w^e_{ij} e_j(\bmx^e_j, \bmp^e_j)
  + \bma_i^T \bmx - b_i.}
as the $i$-th {\em group variable}. If we wish to evaluate
\req{objective}--\req{constraints} and their derivatives,
it thus remains for the user to supply function and derivatives
of the element and group functions to \solver\ as required.
\solver\ allows the user to do this in one of two ways.

The first method simply requires that the user provides
subroutines, with prescribed argument lists, that accept input
values $(\bmx^e_j, \bmp^e_j)$ or $(\alpha^{ }_i, \bmp^g_i)$, and provide
as output
$e_j(\bmx^e_j, \bmp^e_j)$ or $g_i(\alpha^{ }_i, \bmp^g_i)$
or their derivatives. If this method is used, control will only return
to the user once the required minimizer has been obtained (or an error
condition flagged). The disadvantage of this approach is the inflexibility
of the argument lists, while the advantage is often its speed of execution.

The second method is to exit from \solver\ whenever function or derivative
values are required, and simply ask the user to provide
values/derivatives of
$e_j(\bmx^e_j, \bmp^e_j)$ or $g_i(\alpha^{ }_i, \bmp^g_i)$
from given arguments
$(\bmx^e_j, \bmp^e_j)$ or $(\alpha^{ }_i, \bmp^g_i)$ before re-entering.
Here, the advantage is
its flexibility---function/derivative values may be computed by any means,
possibly even by a completely separate calculation---while the
disadvantage is there may be severe performance penalties as
argument values are saved and retrieved on entry/exit from \solver.

The first method is known as {\em internal evaluation}, while the
second is an example of {\em reverse communication}.
Note that we allow the possibility that the user may wish to provide
element values/derivatives using one of the above options,
and group values/derivatives using the other.

%%%%%%%%%%%%%%%%%%%%%% Function types %%%%%%%%%%%%%%%%%%%%%%

\subsubsection{Element and group function types\label{ft}}

The value of the $i$-th element function $e_j$ depends on its {\em elemental}
variables $\bmx_j$ and its parameters $\bmp^e_j$. It is extremely
common in practice that although each element has its own list of
elemental variables and parameters, the functional form of many (if not
all) is identical. For example, suppose that
$\bmx_j = ( x_j )$ and $\bmp^e_j = ( p_j )$,
and that
\disp{e_j(\bmx_j, \bmp^e_j) = p_j x_j^2.}
Then each element $e_j(\bmx_j, \bmp^e_)$ can be expressed as
\disp{e_j(\bmx_j, \bmp^e_j) \equiv e(x_j, p_j)}
of the single element {\em type}
\disp{e(x,p) = p x^2.}
Thus, rather than writing a subroutine to evaluate
the values and derivatives of {\em each} element separately, we simply need to
write a subroutine to evaluate the value/derivative of the {\em single}
element type, and then to evaluate these values/derivatives at each
pair  $(x_i, p_i)$ in turn.

This very simple idea can significantly simplify the coding of
function/derivative evaluation subroutines. In practice, it is unlikely
that each element is of the same type, but usually a few types
suffice to describe all the elements. Thus we provide the flexibility
of allowing the user of specifying how many element types are involved,
and of saying to which type each element belongs. Traditionally,
if there are $n^t_e$ types, they will be numbered from 1 to $n^t_e$,
and there will be a mapping $e^t: \calE \rightarrow \{1,  \ldots, n_e^t \}$
that describes the type for each
element $i \in \calE$. In practice, the types may be numbered in whatever
way best suits the user, so long as distinct types have unique
identifiers---in this case, the set of type identifiers is some $\calE^t$, and
$e^t: \calE \rightarrow \calE^t$.
Moreover, there is nothing to prevent the user from assigning each
element its own type (in which case $n^t_e = n_e$), and setting $e_t(i) = i$.

Precisely the same idea applies to group functions. We will allow the user
to specify how many group types, $n^t_g$ there are, and to provide a
mapping $g^t: \calG \rightarrow \calG^t$, the set of group type identifiers.
It is also convenient to have a special group type,
the {\em trivial} group, for which $g(\alpha) = \alpha$, since
this special type is extremely common---the value/derivative of any group
identified as trivial is indeed so trivial to compute that the user
is not required to do so!

%%%%%%%%%%%%%%%%%%%%%% Range transformations %%%%%%%%%%%%%%%%%%%%%%

\subsubsection{\label{internal_and_ranges}Internal variables and
Subroutine {\tt RANGE}}

A nonlinear element function $e_j$ is assumed to be a function of
the variables $\bmx^e_j$, a subset of the problem variables
$\bmx$. Suppose that $\bmx^e_j$ has $n_j$ elements. Then
another way of saying this is that (ignoring any element parameters)
we have an element function $e_j( v_1, \ldots, v_{n_j})$, where
in our special case, we choose
$v_1= (x^e_j)_1, \ldots, v_{n_j}= (x^e_j)_{n_j}$.
The {\em elemental} variables
for the element function $e_j$ are the variables
$\bar{v}$ and, while we need to associate the particular values
$x^e_j$ with $\bar{v}$ (using the array {\tt IELVAR}),
it is the elemental variables which
are important in defining the nonlinear element functions.

As an example, the seventh nonlinear element function for a
particular problem might be
\eqn{eg1}{e_7(v_1, v_2, v_3 ) = ( v_1 + v_2 ) e^{v_1 - v_3},}
where for our example
$v_1 = x_{29}$, $ v_2 = x_3$ and $v_3 = x_{17}$.
For this example, there are three elemental variables.
However, the
example illustrates an additional point. Although $e_7$ is
a function of three variables, the function itself is really
only composed of {\em two} independent parts; the product of
$(v_1 + v_2)$ with $e^{v_1 - v_3}$, or, if we write
$u_1 = v_1 + v_2$ and $u_2 = v_1 - v_3$, the product of
$u_1$ with $e^{u_2}$.
The variables $u_1$ and $u_2$ are known as {\em internal}
variables
for the element function.
They are obtained as {\em linear combinations} of the elemental variables.
The
important feature as far as \solver\ is concerned is that each
nonlinear function involves as few variables as possible, as
this allows for compact storage and more efficient derivative
approximation. By representing the function in terms of its internal
variables, this goal is achieved.
\solver\ only stores derivatives of the element
functions with respect to internal variables, so it pays to
use an internal representation in this case.
It frequently happens,
however, that a function does not have useful internal variables.
For instance, another element function
might be
\eqn{eg2}{e_9(v_1, v_2) = v_1 \sin v_2,}
where for example $ v_1 = x_6$ and $v_2 = x_{12}$.
Here, we have broken $e_9$ down into as few pieces as possible.
Although there are internal variables,
$u_1 = v_1$ and $u_2 = v_2$,
they are the same in this
case as the elemental variables
and there is no virtue in exploiting them.
Moreover it can happen that although there are
special internal variables,
there are just as many internal as elemental
variables and it therefore doesn't particularly
help to exploit them.
For instance, if
\disp{e_{14}(v_1,v_2) = (v_1+v_2)\log(v_1-v_2),}
where for example
$v_1 = x_{12}$ and $ v_2 = x_2$, the function can be formed as
$u_1 \log(u_2)$ where $u_1 = v_1 + v_2$ and $u_2 = v_1 - v_2$.
But as there are just
as many internal variables as elementals, it will not normally be
advantageous to use this internal representation.

Finally, although an element function
may have  useful internal variables,   the user need  not  bother with
them. \solver\ will still work, but at the expense of extra storage
and computational effort.   The user decides  on input to \solver\
which elements have useful transformations
by setting the appropriate elements of the array {\tt INTREP}  to
\true\ .

In general, there will be a linear transformation from
the elemental variables to the internal ones. For example
(\ref{eg1}), we have
\disp{\vect{u_1 \\ u_2}
  = \mat{rrr}{ 1 & 1 & 0 \\ 1 & 0 & -1} \vect{v_1 \\ v_2 \\ v_3},}
while in (\ref{eg2}), we have
\disp{\vect{u_1 \\ u_2} = \mat{rr}{ 1 & 1 \\ 1 & 0 } \vect{v_1 \\ v_2}.}
Most generally, the transformation will be of the form
\disp{\bmu = \bmW \bmv,}
and this transformation is {\em useful} if the matrix $\bmW$ has
fewer rows than columns.

The purpose of the {\tt RANGE}
routine is to define the transformation
between internal and elemental variables for
nonlinear elements
with useful internal representations.
The routine has the following argument list:

\def\baselinestretch{0.8}
{\tt \begin{verbatim}
     SUBROUTINE RANGE( ielemn, transp, W1, W2, nelvar, ninvar, ieltyp, lw1, lw2 )
\end{verbatim} }
\def\baselinestretch{1.0}

\begin{description}

\itt{ielemn} is a scalar \intentin\ argument of type \integer,
that gives the index of the nonlinear element whose transformation is
required by \solver.

\itt{transp} is a scalar \intentin\ argument of type default \logical.
If {\tt transp} is \false, the subroutine must put the result of the
transformation $\bmW \bmv$ in the array {\tt W2}, where $\bmv$ is input in the
array {\tt W1}. Otherwise, the subroutine must supply the result of the
transposed transformation $\bmW^T \bmu$
in the array {\tt W2}, where $\bmu$ is input in the array {\tt W1}.

\ittf{W1} is a rank-one \intentin array argument of type \realdp\,
whose dimension is the number of elemental variables
if {\tt transp} is \false\
and the number of internal variables otherwise.

\ittf{W2} is a rank-one \intentout array argument of type \realdp\,
whose dimension is
the number of internal variables if {\tt transp} is
\false\ and the number of elemental variables otherwise.
The result of the transformation of {\tt W1} or its transpose, as
defined by {\tt transp}, must be set in {\tt W2}.

\itt{nelvar} is a scalar \intentin\ argument of type \integer,
that gives the number of elemental variables for the element specified
by {\tt ielemn}.

\itt{ninvar} is a scalar \intentin\ argument of type \integer,
which gives the number of internal variables for the element specified
by {\tt ielemn}.

\itt{ieltyp} is a scalar \intentin\ argument of type \integer,
that defines the type for the element specified
by {\tt ielemn}.

\ittf{lw1} is a scalar \intentin\ argument of type \integer,
that will have been set to the actual length of {\tt W1}.

\restriction
{\tt lw1 $\geq$ ninvar} if {\tt transp} is \true\ and
{\tt lw1 $\geq$ nelvar} if {\tt transp} is \false\ .

\ittf{lw2} is a scalar \intentin\ argument of type \integer,
that will have been set to the actual length of {\tt W2}.

\restriction
{\tt lw2 $\geq$ nelvar} if {\tt transp} is \true\ and
{\tt lw2 $\geq$ ninvar} if {\tt transp} is \false\ .

\end{description}

The user will already have specified which elements have useful
transformations in the array {\tt INTREP}. {\tt RANGE} will only be called for
elements for which the corresponding component of {\tt INTREP} is \true\ .

%%%%%%%%%%%%%%%%%%%%%% Element function values %%%%%%%%%%%%%%%%%%%%%%

\subsubsection{Element function values and derivatives via internal evaluation\label{efv}}

If the argument {\tt ELFUN} is present when calling \solver, the
user is expected to provide a subroutine of that name to evaluate
function or derivative values (with respect to internal variables)
of (a given subset of) the element functions.
{\bf N.B.} This routine is only permitted if the argument {\tt ELDERS}
is absent.
The routine has the following argument list:

\def\baselinestretch{0.8}
{\tt \begin{verbatim}
       SUBROUTINE ELFUN ( FUVALS, XVALUE, EPVALU, ncalcf, ITYPEE, ISTAEV, IELVAR, INTVAR, &
                          ISTADH, ISTEPA, ICALCF, ltypee, lstaev, lelvar, lntvar, lstadh, &
                          lstepa, lcalcf, lfuval, lxvalu, lepvlu, ifflag, ifstat )
\end{verbatim} }
\def\baselinestretch{1.0}

\begin{description}

\itt{FUVALS} is a rank-one \intentinout array argument of
dimension {\tt lfuval} and type \realdp\,
which is used to store
function and derivative information for the nonlinear element functions.
The subroutine is asked to provide values for these functions
and/or derivatives,
evaluated at
the argument {\tt XVALUE} (see {\tt XVALUE}),
at specified locations within {\tt FUVALS}.
The layout of {\tt FUVALS} is indicated in Table~\ref{2.4b}.

{\scriptsize
\bctable{|c|c|c|l}
\multicolumn{1}{c}{$\longleftarrow$~~~$n_e$~~~$\longrightarrow$} &
\multicolumn{1}{c}{} &
\multicolumn{1}{c}{} &
\\
\cline{1-3}
element & \hspace*{1mm} element \hspace*{1mm}   &
\hspace*{1mm} element \hspace*{1mm}  \\
values  & gradients & Hessians \\
\cline{1-3}
\multicolumn{1}{l}{} &
\multicolumn{1}{l}{\hspace*{-2mm}$\uparrow$} &
\multicolumn{1}{l}{\hspace*{-2mm}$\uparrow$} &
\multicolumn{1}{l}{\hspace*{-2mm}$\uparrow$} \\
\multicolumn{1}{l}{} &
\multicolumn{1}{l}{\hspace*{-8mm}\tt INTVAR(1)} &
\multicolumn{1}{l}{\hspace*{-8mm}\tt ISTADH(1)} &
\multicolumn{1}{l}{\hspace*{-8mm}\tt ISTADH($n_e+1$)}
\ectable{\label{2.4b}Partitioning of the workspace array {\tt FUVALS}}
}
\vspace*{-4mm}

The first segment of {\tt FUVALS} is used to hold the values of the nonlinear
element functions, component $i$ holding the value of the $i$-th element
function. The second segment holds the components of the
gradients of the element functions
taken with respect to their internal variables, as described by
{\tt INTVAR} below. The final segment contains the elements'
Hessian matrices, again taken with respect to their internal variables,
as described by {\tt ISTADH} below.

\itt{XVALUE} is a rank-one \intentin array argument of
dimension {\tt lxvalu} and type \realdp\,
that contains the values of $\bmx$ at which the
subroutine is required to evaluate the values or derivatives
of the nonlinear elements functions.

\itt{EPVALU} is a rank-one \intentin array argument of
dimension {\tt lepvlu}  and type \realdp\,
that contains the values of the element parameters $\bmp^e_i$, $i = 1, \ldots,
n_e$. The indices for the parameters for element $i$ immediately precede
those for element $i+1$, and each element's parameters appear in a contiguous
list. See Table~\ref{3.2a} for an illustration.

\itt{ncalcf} is a scalar \intentin\ argument of type \integer,
that specifies how many of the nonlinear element functions or their
derivatives are to be evaluated.

\itt{ITYPEE} is a rank-one \intentin array argument of
dimension {\tt ltypee} and type \integer\,
whose $i$-th component specifies the type of element $i$.

\itt{ISTAEV} is a rank-one \intentin array argument of
dimension {\tt lstaev} and type \integer\,
exactly as described in Section~\ref{typeprob}.

\itt{IELVAR} is a rank-one \intentin array argument of
dimension {\tt lelvar} and type \integer\,
exactly as described in Section~\ref{typeprob}.

\itt{INTVAR} is a rank-one \intentin array argument of
dimension {\tt lntvar} and type \integer\,
whose $i$-th component $(1 \leq i \leq n_e )$
gives the position in the array {\tt FUVALS} of the first component of
the gradient of the $i$-th nonlinear element function, {\em with respect
to its internal variables}.
See Table~\ref{3.3a} for an illustration.

\bctable{cc|c|c|c|cc}
\cline{1-6}
\multicolumn{1}{c|}{.}
& gradient of $e_1$  wrt & gradient of $e_2$ wrt
& \hspace*{5mm} \ldots \hspace*{5mm} &
gradient of $e_{n_e}$ wrt & . & {\tt FUVALS} \\
\multicolumn{1}{c|}{} & its internal variables & its internal variables &
& its internal variables &  \\
\cline{1-6}
& \multicolumn{1}{l}{\hspace*{-2mm}$\uparrow$} &
\multicolumn{1}{l}{\hspace*{-2mm}$\uparrow$} &
\multicolumn{1}{l}{\hspace*{-2mm}$\uparrow$} &
\multicolumn{1}{l}{\hspace*{-2mm}$\uparrow$} &
\multicolumn{1}{l}{\hspace*{-2mm}$\uparrow$} & \\
&
\multicolumn{1}{l}{\hspace*{-8mm}\tt INTVAR(1)} &
\multicolumn{1}{l}{\hspace*{-8mm}\tt INTVAR(2)} &
\multicolumn{1}{l}{\hspace*{-8mm}\tt INTVAR(3)} &
\multicolumn{1}{l}{\hspace*{-9mm}\tt INTVAR($n_e$)} & &
\multicolumn{1}{l}{\hspace*{-15mm}\tt INTVAR($n_e+1$)} \\
& \multicolumn{1}{l}{} & \multicolumn{1}{l}{} & \multicolumn{1}{l}{} &
\multicolumn{1}{l}{} & \multicolumn{1}{l}{} &
\multicolumn{1}{l}{\hspace*{-15mm}\tt $\equiv$ ISTADH(1) }
\ectable{\label{3.3a}Contents of the arrays {\tt INTVAR} and (part of)
{\tt FUVALS}}
\vspace*{-4mm}

\itt{ISTADH} is a rank-one \intentin array argument of
dimension {\tt lstadh} and type \integer\,
whose $i$-th component
$(1 \leq i \leq n_e )$ gives the position in the array {\tt FUVALS} of
the first component of the Hessian matrix of the $i$-th nonlinear
element function $e_i$, {\em with respect to its internal variables}.
Only the upper triangular part of each Hessian matrix is stored and the
storage is by columns. That is to say that
the component of the Hessian of the $k$-th nonlinear element
with respect to internal variables
$i$ and $j$, $i \leq j$,
\disp{\frac{\partial^2 e_k}{\partial u_i \partial u_j },}
must be  placed in {\tt FUVALS(ISTADH($k$)+$(j(j-1)/2)+i-1$)}.
The element {\tt ISTADH($n_e +1$)} is space required
to finish storing the  Hessian of the last nonlinear element in {\tt FUVALS}
plus one.
See Table~\ref{3.4a} for an illustration.

\bctable{cc|c|c|c|c}
\cline{1-5}
\multicolumn{1}{c|}{.}
& Hessian of $e_1$  wrt & Hessian of $e_2$ wrt
& \hspace*{5mm} \ldots \hspace*{5mm} &
Hessian of $e_{n_e}$ wrt & {\tt FUVALS} \\
\multicolumn{1}{c|}{} & its internal variables & its internal variables &
& its internal variables &  \\
\cline{1-5}
& \multicolumn{1}{l}{\hspace*{-2mm}$\uparrow$} &
\multicolumn{1}{l}{\hspace*{-2mm}$\uparrow$} &
\multicolumn{1}{l}{\hspace*{-2mm}$\uparrow$} &
\multicolumn{1}{l}{\hspace*{-2mm}$\uparrow$} &
\multicolumn{1}{l}{\hspace*{-2mm}$\uparrow$} \\
& \multicolumn{1}{l}{\hspace*{-8mm}\tt ISTADH(1)} &
\multicolumn{1}{l}{\hspace*{-8mm}\tt ISTADH(2)} &
\multicolumn{1}{l}{\hspace*{-8mm}\tt ISTADH(3)} &
\multicolumn{1}{l}{\hspace*{-9mm}\tt ISTADH($n_e$)} &
\multicolumn{1}{l}{\hspace*{-15mm}\tt ISTADH($n_e+1$)}
\ectable{\label{3.4a}Contents of the arrays {\tt ISTADH} and (part of)
{\tt FUVALS}}

\itt{ISTEPA} is a rank-one \intentin array argument of
dimension {\tt lstepa} and type \integer\,
whose $i$-th component gives the position in
{\tt EPVALU} of the first parameter for element function $i$.  In
addition, {\tt ISTEPA($n_e$+1)} is the position in
{\tt EPVALU} of the last  parameter for element function $n_e$ plus one. See
Table~\ref{3.2a} for an illustration.

\itt{ICALCF} is a rank-one \intentin array argument of
dimension {\tt lcalcf} and type \integer\,
whose first {\tt ncalcf} components
gives the indices of the nonlinear element functions whose values or
derivatives are to be evaluated.

\itt{ltypee$,$} \hspace*{-2mm}
{\tt lstaev},
{\tt lelvar},
{\tt lntvar},
{\tt lstadh},
{\tt lstepa},
{\tt lcalcf},
{\tt lfuval},
{\tt lxvalu} and
{\tt lepvlu}
are all scalar \intentin\ arguments of type \integer.
They will have been set to the actual lengths of
{\tt ITYPEE},
{\tt ISTAEV},
{\tt IELVAR},
{\tt INTVAR},
{\tt ISTADH},
{\tt ISTEPA},
{\tt ICALCF},
{\tt FUVALS},
{\tt XVALUE} and
{\tt EPVALU} respectively.

\restrictions
{\tt ltypee $\geq$ prob\%nel},
{\tt lstaev $\geq$ prob\%nel+1},
{\tt lelvar $\geq$ ISTAEV(prob\%nel+1)-1}, \\
{\tt lntvar $\geq$ prob\%nel+1},
{\tt lstadh $\geq$ prob\%nel+1},
{\tt lstepa $\geq$ prob\%nel+1},
{\tt lcalcf $\geq$ ncalcf}, \\
{\tt lfuval $\geq$ ISTADH(prob\%nel+1)-1},
{\tt lxvalu $\geq$ prob\%n}, and
{\tt lepvlu $\geq$ ISTEPA(prob\%nel+1)-1}.

\itt{ifflag} is a scalar \intentin\ argument of type \integer,
whose value defines whether it is the values of the element functions
that are required ({\tt ifflag = 1}) or if it is the derivatives
({\tt ifflag $>$ 1}). Possible values and their requirements are:
\begin{description}

\itt{ifflag = 1$\mbox{.}$} The values of nonlinear element functions
{\tt ICALCF($i$)}, $i = ${\tt 1}$, \ldots ,$ {\tt ncalcf}, are to be computed
and placed in {\tt FUVALS(ICALCF($i$))}.

\itt{ifflag = 2$\mbox{.}$} The gradients of nonlinear element functions
{\tt ICALCF($i$)}, $i = ${\tt 1}$, \ldots ,$ {\tt ncalcf}, are to be computed.
The gradient of the {\tt ICALCF($i$)}-th element is to be placed in
the segment of {\tt FUVALS} starting at {\tt INTVAR(ICALCF($i$))}
(see Table~\ref{3.3a}).

\itt{ifflag = 3$\mbox{.}$} The gradients and Hessians
of nonlinear element functions
{\tt ICALCF($i$)}, $i = ${\tt 1}$, \ldots ,$ {\tt ncalcf}, are to be computed.
The gradient of the {\tt ICALCF($i$)}-th element is to be placed in
the segment of {\tt FUVALS} starting at {\tt INTVAR(ICALCF($i$))}
(see Table~\ref{3.3a}).
The Hessian of this element should be placed in
the segment of {\tt FUVALS} starting at {\tt ISTADH(ICALCF($i$))}
(see Table~\ref{3.4a}).

\end{description}
{\bf N.B.} If the user intends to use approximate second derivatives
({\tt control\%second\_derivatives $>$ 0}, see Section~\ref{typecontrol}),
\solver\ will never call {\tt ELFUN} with {\tt ifflag = 3}, so the
user need not provide Hessian values. Furthermore, if the user
intends to use approximate first derivatives
({\tt control\%first\_derivatives $>$ 0}, see Section~\ref{typecontrol}),
\solver\ will never call {\tt ELFUN} with {\tt ifflag = 2} or {\tt 3}, so the
user need then not provide gradient or Hessian values.

\itt{ifstat} is a scalar \intentout\ argument of type \integer,
that should be set to 0 if all of the required values have been found,
and to any nonzero value if, for any reason, one or more of the required
values could not be determined. For instance, if the value of a
nonlinear element (or its derivative) was required outside of its domain of
definition, a nonzero value of {\tt ifstat} should be returned.

\end{description}


%%%%%%%%%%%%%%%%%%%%%% Element function values %%%%%%%%%%%%%%%%%%%%%%

\subsubsection{Element function values and flexible derivatives via internal evaluation\label{efvf}}

If the argument {\tt ELFUN\_flexible} is present when calling \solver, the
user is expected to provide a subroutine of that name to evaluate
function or derivative values (with respect to internal variables)
of (a given subset of) the element functions.
{\bf N.B.} This routine is only permitted if the argument {\tt ELDERS}
is also present.
The routine has the following argument list:

\def\baselinestretch{0.8}
{\tt \begin{verbatim}
       SUBROUTINE ELFUN_flexible ( FUVALS, XVALUE, EPVALU, ncalcf, ITYPEE, ISTAEV, IELVAR, &
                                   INTVAR, ISTADH, ISTEPA, ICALCF, ltypee, lstaev, lelvar, &
                                   lntvar, lstadh, lstepa, lcalcf, lfuval, lxvalu, lepvlu, &
                                   llders, ifflag, ELDERS, ifstat )
\end{verbatim} }
\def\baselinestretch{1.0}

\begin{description}

\itt{FUVALS\rm ,}
{\tt FUVALS}, {\tt XVALUE}, {\tt EPVALU}, {\tt ncalcf}, {\tt ITYPEE},
{\tt ISTAEV}, {\tt IELVAR},
{\tt INTVAR}, {\tt ISTADH}, {\tt ISTEPA}, {\tt ICALCF}, {\tt ltypee},
{\tt lstaev}, {\tt lelvar},
{\tt lntvar}, {\tt lstadh}, {\tt lstepa}, {\tt lcalcf}, {\tt lfuval},
{\tt lxvalu}, {\tt lepvlu} and
{\tt ifstat} are all exactly as described for subroutine {\tt ELFUN}
in Section~\ref{efv}

\itt{llders} is a scalar \intentin\ argument of type \integer, that
will have been set to the actual length of {\tt ELDERS}.

\restriction
{\tt llders $\geq$ prob\%nel}.

\itt{ifflag} is a scalar \intentin\ argument of type \integer,
whose value defines whether it is the values of the element functions
that are required ({\tt ifflag = 1}) or if it is the derivatives
({\tt ifflag $>$ 1}). If
{\tt ifflag = 1}, the values of nonlinear element functions
{\tt ICALCF($i$)}, $i = ${\tt 1}$, \ldots ,$ {\tt ncalcf}, are to be computed
and placed in {\tt FUVALS(ICALCF($i$))}. If {\tt ifflag $>$ 1}, those
derivative values specified by {\tt ELDERS} (see below) are required.


\itt{ELDERS}
is an rank-two \intentinout array argument of
shape {\tt (2, llders)} and type \integer\,
that specifies what first- and second-derivative information (if any) is
required for each nonlinear element function when {\tt ifflag $>$ 1}.
Specifically, in this case, the gradient of each nonlinear element function
{\tt ICALCF($i$)}, $i = ${\tt 1}$, \ldots ,$ {\tt ncalcf},
for which {\tt ELDERS(1,ICALCF($i$)) $\leq$ 0} should be computed.
The gradient of the {\tt ICALCF($i$)}-th element is to be placed in
the segment of {\tt FUVALS} starting at {\tt INTVAR(ICALCF($i$))}
(see Table~\ref{3.3a}). Furthermore, if additionally
{\tt ELDERS(2,ICALCF($i$)) $\leq$ 0}, the Hessian of the
nonlinear element function {\tt ICALCF($i$)} should be computed and placed in
the segment of {\tt FUVALS} starting at {\tt ISTADH(ICALCF($i$))}
(see Table~\ref{3.4a}).

\end{description}

%%%%%%%%%%%%%%%%%%%%%% Group function values %%%%%%%%%%%%%%%%%%%%%%

\subsubsection{Group function values and derivatives via internal
evaluation\label{gfv}}

If the argument {\tt GROUP} is present when calling \solver, the
user is expected to provide a subroutine of that name to evaluate
function or derivative values of (a given subset of) the group functions.
The routine has the following argument list:

\def\baselinestretch{0.8}
{\tt \begin{verbatim}
       SUBROUTINE GROUP ( GVALUE, lgvalu, FVALUE, GPVALU, ncalcg, ITYPEG, ISTGPA, ICALCG, &
                          ltypeg, lstgpa, lcalcg, lfvalu, lgpvlu, derivs, igstat )
\end{verbatim} }
\def\baselinestretch{1.0}

\begin{description}

\itt{GVALUE} is a rank-two \intentout array argument of
shape {\tt (lgvalu, 3)} and type \realdp.
The value and first and second derivative of the $i$-th group
function are held in
{\tt GVALUE($i$,1)},
{\tt GVALUE($i$,2)} and
{\tt GVALUE($i$,3)}
respectively.

\itt{lgvalu} is a scalar \intentin\ argument of type \integer,
that will have been set to the actual length of the leading dimension
of {\tt GVALUE}.

\restriction
{\tt lgvalu $\geq$ prob\%ng}.

\itt{FVALUE} is a rank-one \intentin array argument of
dimension {\tt lfvalu} and type \realdp\,
whose $i$-th component contains the value of group variable at which the
subroutine is required to evaluate the value or derivatives
of the $i$-th group function.

\itt{GPVALU} is a rank-one \intentin array argument of
dimension {\tt lgpvlu}  and type \realdp\,
that contains the values of the group parameters $\bmp^g_i$, $i = 1, \ldots,
n_g$. The indices for the parameters for group $i$ immediately precede
those for group $i+1$, and each group's parameters appear in a contiguous
list. See Table~\ref{4.2a} for an illustration.

\itt{ncalcg} is a scalar \intentin\ argument of type \integer,
that specifies how many of the group functions or their
derivatives are to be evaluated.

\itt{ITYPEG} is a rank-one \intentin array argument of
dimension {\tt ltypeg} and type \integer\,
whose $i$-th component specifies the type of group $i$.

\itt{ISTGPA} is a rank-one \intentin array argument of
dimension {\tt lstgpa} and type \integer\,
whose $i$-th component gives the position in
{\tt GPVALU} of the first parameter for group function $i$.  In
addition, {\tt ISTGPA($n_g$+1)} is the position in
{\tt GPVALU} of the last  parameter for element function $n_g$ plus one. See
Table~\ref{4.2a} for an illustration.

\itt{ICALCG} is a rank-one \intentin array argument of
dimension {\tt lcalcg} and type \integer\,
whose first {\tt ncalcg} components
gives the indices of the nonlinear group functions whose values or
derivatives are to be evaluated.

\itt{ltypeg$,$} \hspace*{-2mm}
{\tt lstgpa},
{\tt lcalcg},
{\tt lfvalu} and
{\tt lgpvlu}
are all scalar \intentin\ arguments of type \integer.
They will have been set to the actual lengths of
{\tt ITYPEG},
{\tt ISTGPA},
{\tt ICALCG},
{\tt FVALUE} and
{\tt GPVALU}.

\restrictions
{\tt ltypeg $\geq$  prob\%n},
{\tt lstgpa $\geq$ prob\%ng+1},
{\tt lcalcg $\geq$ ncalcg},
{\tt lfvalu $\geq$ prob\%ng}, and
{\tt lgpvlu $\geq$ ISTGPA(prob\%ng+1)-1}.

\itt{derivs} is a scalar \intentin\ argument of type default \logical.
When {\tt derivs} is \false, the subroutine must return
the values of group functions
{\tt ICALCG($i$)}, $i = ${\tt 1}$, \ldots ,$ {\tt ncalcg} in
{\tt GVALUE(ICALCF($i$),1)}.
When {\tt derivs} is \true, the subroutine must return
the first and second derivatives of group functions
{\tt ICALCG($i$)}, $i = ${\tt 1}$, \ldots ,$ {\tt ncalcg} in
{\tt GVALUE(ICALCF($i$),2)}
and
{\tt GVALUE(ICALCF($i$),3)}
respectively.

\itt{igstat} is a scalar \intentout\ argument of type \integer,
that should be set to 0 if all of the required values have been found,
and to any nonzero value if, for any reason, one or more of the required
values could not be determined. For instance, if the value of a
group function (or its derivative) was required outside of its domain of
definition, a nonzero value of {\tt igstat} should be returned.

\end{description}

\subsubsection{\label{reverse}Reverse Communication Information}

When a return is made from \solver\
with {\tt inform\%status} set negative,
\solver\
is asking the user for further information---this will happen
if the user has chosen not to evaluate element or group values
internally (see Section~\ref{fdv}), or if the user wishes to provide
his or her own preconditioner. The user
should normally compute the required information and re-enter \solver\
with {\tt inform\%status} unchanged
(but see {\tt inform\%status} $= -11$, below, for an exception).

Possible values of {\tt inform\%status} and the information required are
\begin{description}
\itt{inform\%status $= -1.$} The user should compute the function
     and, if they are available,
     derivative values of the nonlinear element functions
     numbered {\tt ICALCF($i$)} for $i=1, \ldots,$ {\tt ncalcf}
     (see below).

\itt{inform\%status $= -2.$} The user should compute the function
     and derivative values of all non-trivial nonlinear group functions,
     numbered {\tt ICALCG($i$)} for $i=1, \ldots,$ {\tt ncalcg}
     (see below).

\itt{inform\%status $= -3.$} The user should compute the
     function values
     of the nonlinear element functions
     numbered {\tt ICALCF($i$)} for $i=1, \ldots,$ {\tt ncalcf}
     (see below).

\itt{inform\%status $= -4.$} The user should compute the function
     values of all non-trivial nonlinear group  functions
     numbered {\tt ICALCG($i$)} for $i=1, \ldots,$ {\tt ncalcg}
     (see below).

\itt{inform\%status $= -5.$} If exact derivatives are available,
     the user should compute the
     derivative values of all nonlinear element functions
     numbered {\tt ICALCF($i$)} for $i=1, \ldots,$ {\tt ncalcf}.
     The user should also compute the derivative
     values of all non-trivial nonlinear group  functions
     numbered {\tt ICALCG($i$)} for $i=1, \ldots,$ {\tt ncalcg}
     (see below).

\itt{inform\%status $= -6.$} The user should compute the
     derivative values of all nonlinear element functions
     numbered {\tt ICALCF($i$)} for $i=1, \ldots,$ {\tt ncalcf}
     (see below).

\itt{inform\%status $= -7.$} The user should compute the
     function values
     of the nonlinear element functions
     numbered {\tt ICALCF($i$)} for $i=1, \ldots,$ {\tt ncalcf}
     (see below).

\itt{inform\%status $= -8, -9, -10.$} The user should compute the
     product $\bmq = \bmM\bmd$ of a preconditioning matrix $\bmM$
     with the vector $\bmd$ and return the value in $\bmq$
     (see below).
\end{description}

When {\tt inform\%status} $= -1$, $-3$ or $-7$, the user must return the
values of all
of the nonlinear element functions
$e_k(x^e_k)$,
$k = ${\tt ICALCF($i$)}, $i= 1, \ldots,$ {\tt ncalcf}.
The functions are to be evaluated at the point $x$ given
in the array {\tt XT}.
The $k$-th function value must be placed in {\tt FUVALS}(k).

When {\tt inform\%status} $= -1, -5$ or $-6$
and all first derivatives are available
(that is when {\tt control\%first\_der\-ivat\-ives = 0}),
the user must return the values
of the gradients, {\em with respect to their internal variables}, of all
the nonlinear element functions
$e_k(\bmx^e_k,\bmp^e_k)$,
$k = ${\tt  ICALCF($i$)}, $i=1,  \ldots,$ {\tt ncalcf}---if a subset of the
gradients are available (those for which {\tt ELDERS(1,$k$) $\leq$ 0}),
only these need be provided.
The gradients are to be evaluated at the point  $\bmx$ given in
the array  {\tt XT}.  The  gradient with respect  to internal variable
$i$ of the $k$-th nonlinear element,
\disp{\frac{\partial e_k}{\partial u_i},}
must be placed in {\tt FUVALS(INTVAR($k$)+$i$-1)}.
If, in addition, exact second derivatives
are to be provided
({\tt control\%se\-cond\_derivatives = 0}), the user must return the values
of the Hessian matrices, {\em with respect to their internal variables},
of the same nonlinear element functions
evaluated at $\bmx$---again, if a subset of the
Hessians are available (those for which {\tt ELDERS(2,$k$) $\leq$ 0}),
only these need be provided.
Only the ``upper triangular'' part of the
required Hessians should be specified.
The component of the Hessian of the $k$-th nonlinear element
with respect to internal variables
$i$ and $j$, $i \leq j$,
\disp{\frac{\partial^2 e_k}{\partial u_i \partial u_j },}
must be  placed in {\tt FUVALS(ISTADH($k$)+$(j(j-1)/2)+i-1$)}.

When {\tt inform\%status} $ = -2$ or $- 4$, the user must return the values
of all the group functions
$g_k$, $k  = ${\tt ICALCG($i$)},   $i=1,  \ldots,$ {\tt ncalcg}.   The
$k$-th such  function  should  be  evaluated  with the argument
{\tt FT($k$)} and the result placed in {\tt GVALS($k$,1)}.

When {\tt inform\%status} $= -2$ or $-5$, the user must return the values
of the first and second derivatives of each of
the group functions
$g_k$, $k =  ${\tt   ICALCG($i$)}, $i=1, \ldots,$ {\tt  ncalcg},  with
respect to its argument.  The derivatives of the  $k$-th such function
should be  evaluated with   the   argument  {\tt FT($k$)}.  The  first
derivative
of the $k$-th group function should be placed in {\tt GVALS($k$,2)} and the
corresponding second derivative returned in {\tt GVALS($k$,3)}.

When {\tt inform\%status} $= -8, -9$ or $- 10$, the user must return the
values of the components {\tt Q(IVAR($i$))}
$= q_i$, $i = 1, \ldots,$ {\tt inform\%nvar}, such
that $\bmq=\bmM \bmd$ and where $d_i=${\tt DGRAD($i$)} $i=1, \ldots,$
{\tt inform\%nvar}. Here $\bmM$ is a symmetric positive definite
approximation to the inverse of the matrix
whose $i$-th row and column are the
{\tt IVAR($i$)}-th row and column
of the Hessian matrix of augmented Lagrangian function (see
section~\ref{galmethod}),
$i = 1, \ldots,$ {\tt inform\%nvar}.
These values can only occur if {\tt control\%linear\_solver = 3}.

If the user does not wish, or is unable, to compute an element or
group function
at a particular argument returned from \solver,
{\tt inform\%status} may be reset to -11 and \solver\ re-entered. \solver\
will treat such a re-entry as if the current iteration had been
unsuccessful and reduce the trust-region radius. This facility is
useful when, for instance, the user is asked to evaluate a function
at a point outside its domain of definition.

%%%%%%%%%%%%%%%%%%%%%% Warning and error messages %%%%%%%%%%%%%%%%%%%%%%


\galerrors
If {\tt inform\%status} is positive
on return from \solver, an error has been detected.
The user should correct the error and restart the minimization.
Possible values of {\tt inform\%status} and their consequences are:
\begin{description}
\itt{inform\%status $= 1.$} More than {\tt control\%maxit} iterations have been
      performed. This is often a symptom of incorrectly programmed
      derivatives or of the preconditioner used being
      insufficiently effective. Recheck the derivatives.
      Otherwise, increase {\tt control\%maxit} and re-enter \solver\
      at the best point found so far.
\itt{inform\%status $= 2.$} The trust-region radius
      has become too small. This is
      often a symptom of incorrectly programmed derivatives
      or of requesting more accuracy in the projected gradient
      than is reasonable on the user's machine.
      If the projected gradient
      is small, the minimization
      has probably succeeded. Otherwise, recheck the derivatives.
\itt{inform\%status $= 3.$} The step taken during the current iteration is so
      small that
      no difference will be observed in the function values.
      This sometimes occurs when too much accuracy is required of
      the final gradient. If the projected gradient
      is small, the minimization has probably succeeded.
\itt{inform\%status $= 4.$} One of the {\tt INTEGER} arrays has been
      initialized with insufficient space. A message indicating which array is
      at fault and the required space
      will be printed on unit number {\tt control\%error}.
\itt{inform\%status $= 5.$} One of the {\tt REAL} ({\tt DOUBLE PRECISION} in
      the D version) arrays has been initialized with
      insufficient space. A message indicating which array is
      at fault and the required space
      will be printed on unit number {\tt control\%error}.
\itt{inform\%status $= 6.$} One of the {\tt LOGICAL} arrays has been initialized with
      insufficient space. A message indicating which array is
      at fault and the required space
      will be printed on unit number {\tt control\%error}.
\itt{inform\%status $= 7.$} One or more of the components of the array
      {\tt KNDOFG} does not have the value 0, 1 or 2.
\itt{inform\%status $= 8.$} The problem does not appear to have a feasible
     solution. Check the constraints and try starting with different
     initial values for $\bmx$.
\itt{inform\%status $= 9.$} One of the arrays
     {\tt prob\%KNDOFG}, {\tt prob\%C} or {\tt prob\%Y}
     is either not associated or is not large enough.
     (Re-)allocate the arrays to have dimensions at least prob\%ng.
\itt{inform\%status $= 10.$} The user has provided {\tt ELFUN} to
      perform internal element evaluation, but one or both of
      {\tt prob\%ISTEPA} and {\tt prob\%EPVALU} is not associated
      or at least one is not large enough. (Re-)allocate (and reset)
      {\tt prob\%ISTEPA} to have dimension at least {\tt prob\%nel+1}
      and {\tt prob\%EPVALU} to have dimension at least
      {\tt prob\%ISTEPA(prob\%nel+1)-1}.
\itt{inform\%status $= 11.$} The user has provided {\tt GROUP} to
      perform internal group evaluation, but one or both of
      {\tt prob\-\%ISTGPA} and {\tt prob\%GPVALU} is not associated
      or at least one is not large enough. (Re-)allocate (and reset)
      {\tt prob\%ISTGPA} to have dimension at least {\tt prob\%ng+1}
      and {\tt prob\%GPVALU} to have dimension at least
      {\tt prob\%ISTG\-PA(prob\%ng+1)-1}.
\itt{inform\%status $= 12.$} An internal array allocation or de-allocation
      has failed. The name of the offending array is given in
      {\tt inform\%bad\_alloc}, and the allocation status is
      {\tt inform\%alloc\_status}.
\itt{inform\%status $= 13.$} One or more of the problem functions
      cannot be computed at the initial point.
      Try starting with a different initial value for $\bmx$.
\itt{inform\%status $= 14.$} The user has forced termination of \solver\
     by removing the file named {\tt control\%a\-live\_file} from unit
     unit {\tt control\%alive\_unit}.
\itt{inform\%status $= 15.$} One or more of the problem dimensions
     {\tt prob\%n}, {\tt prob\%ng} or {\tt prob\%nel} is too small. Ensure that
     {\tt prob\%n $>$ 0}, {\tt prob\%ng $>$ 0} or {\tt prob\%nel $\geq$ 0}.
\itt{inform\%status $= 16.$} Both optional arguments {\tt ELDERS}
    and {\tt ELFUN} are present at the same time. Either remove {\tt ELDERS}
     (or {\tt ELFUN}) or replace {\tt ELFUN} by {\tt ELFUN\_flexible}.
\itt{inform\%status $= 17.$} The optional argument {\tt ELFUN\_flexible}
    is present, but {\tt ELDERS} is absent.
    Either include {\tt ELDERS} or remove {\tt ELFUN\_flexible}.
\itt{inform\%status $= 18.$} The current value of the merit function
  \req{objectiveb}, {\tt inform\%aug}, is smaller than its smallest
  permitted value, {\tt control\%min\_aug}. Check the formulation to see
  if this seems reasonable.
\itt{inform\%status $= 19.$} The maximum CPU time limit has been exceed.
\itt{inform\%status $= 26.$} A requested linear solver is unavailable.
\end{description}

%%%%%%%%%%%%%%%%%%%%%% Further features %%%%%%%%%%%%%%%%%%%%%%%%

\galfeatures
\noindent In this section, we describe an alternative means of setting
control parameters, that is components of the variable {\tt control} of type
{\tt \packagename\_control\_type}
(see Section~\ref{typecontrol}),
by reading an appropriate data specification file using the
subroutine {\tt \packagename\_read\_specfile}. This facility is useful
as it allows a user to change  {\tt \packagename} control parameters
without editing and recompiling programs that call {\tt \packagename}.

A specification file, or specfile, is a data file containing a number of
"specification commands". Each command occurs on a separate line,
and comprises a "keyword",
which is a string (in a close-to-natural language) used to identify a
control parameter, and
an (optional) "value", which defines the value to be assigned to the given
control parameter. All keywords and values are case insensitive,
keywords may be preceded by one or more blanks but
values must not contain blanks, and
each value must be separated from its keyword by at least one blank.
Values must not contain more than 30 characters, and
each line of the specfile is limited to 80 characters,
including the blanks separating keyword and value.

The portion of the specification file used by
{\tt \packagename\_read\_specfile}
must start
with a "{\tt BEGIN \packagename}" command and end with an
"{\tt END}" command.  The syntax of the specfile is thus defined as follows:
\begin{verbatim}
  ( .. lines ignored by LANCELOT_read_specfile .. )
    BEGIN LANCELOT
       keyword    value
       .......    .....
       keyword    value
    END
  ( .. lines ignored by LANCELOT_read_specfile .. )
\end{verbatim}
where keyword and value are two strings separated by (at least) one blank.
The ``{\tt BEGIN \packagename}'' and ``{\tt END}'' delimiter command lines
may contain additional (trailing) strings so long as such strings are
separated by one or more blanks, so that lines such as
\begin{verbatim}
    BEGIN LANCELOT SPECIFICATION
\end{verbatim}
and
\begin{verbatim}
    END LANCELOT SPECIFICATION
\end{verbatim}
are acceptable. Furthermore,
between the
``{\tt BEGIN \packagename}'' and ``{\tt END}'' delimiters,
specification commands may occur in any order.  Blank lines and
lines whose first non-blank character is {\tt !} or {\tt *} are ignored.
The content
of a line after a {\tt !} or {\tt *} character is also
ignored (as is the {\tt !} or {\tt *}
character itself). This provides an easy manner to "comment out" some
specification commands, or to comment specific values
of certain control parameters.

The value of a control parameters may be of five different types, namely
integer, logical, real, character string, or symbolic.
Integer and real values may be expressed in any relevant Fortran integer and
floating-point formats (respectively), while
character strings are arbitrary collections of default Fortran character
constants (without quotation marks).
A symbolic value is a special string (defined in the
{\tt GALAHAD\_SYMBOLS module}),
that may help to express a control parameter for {\tt \packagename}
in a "language" that is close to natural.  Permitted values for logical
parameters are "{\tt ON}", "{\tt TRUE}", "{\tt .TRUE.}", "{\tt T}",
"{\tt YES}", "{\tt Y}", or "{\tt OFF}", "{\tt NO}",
"{\tt N}", "{\tt FALSE}", "{\tt .FALSE.}" and "{\tt F}".
Empty values are also allowed for
logical control parameters, and are interpreted as "{\tt TRUE}".

The specification file must be open for
input when {\tt \packagename\_read\_specfile}
is called, and the associated device number
passed to the routine in device (see below).
Note that the corresponding
file is {\tt REWIND}ed, which makes it possible to combine the specifications
for more than one program/routine.  For the same reason, the file is not
closed by {\tt \packagename\_read\_specfile}.

\subsubsection{To read control parameters from a specification file}
\label{readspec}

Control parameters may be read from a file as follows:
\hskip0.5in
\def\baselinestretch{0.8} {\tt \begin{verbatim}
     CALL LANCELOT_read_specfile( control, device )
\end{verbatim}}
\def\baselinestretch{1.0}

\begin{description}
\itt{control} is a scalar \intentinout argument of type
{\tt \packagename\_control\_type}
(see Section~\ref{typecontrol}).
Default values should have already have set, perhaps by calling
{\tt \packagename\_initialize}.
On exit, individual components of {\tt control} may have been changed
according to the commands found in the specfile. Specfile commands and
the component (see Section~\ref{typecontrol}) of {\tt control}
that each affects are given in Table~\ref{specfile}; permitted string
values are described in Table~\ref{string}.

\pctable{|l|l|l|}
\hline
  command & component of {\tt control} & value type \\
\hline
  {\tt error-printout-device} & {\tt \%error} & integer \\
  {\tt printout-device} & {\tt \%out} & integer \\
  {\tt alive-device} & {\tt \%alive\_unit} & integer \\
  {\tt print-level} & {\tt \%print\_level} & integer \\
  {\tt maximum-number-of-iterations} & {\tt \%maxit} & integer \\
  {\tt start-print} & {\tt \%start\_print} & integer \\
  {\tt stop-print} & {\tt \%stop\_print} & integer \\
  {\tt iterations-between-printing} & {\tt \%print\_gap} & integer \\
  {\tt linear-solver-used} & {\tt \%linear\_solver} & string \\
  {\tt number-of-lin-more-vectors-used} & {\tt \%icfact} & integer \\
  {\tt semi-bandwidth-for-band-preconditioner} & {\tt \%semibandwidth} & integer\\
  {\tt maximum-dimension-of-schur-complement} & {\tt \%max\_sc} & integer \\
  {\tt unit-number-for-temporary-io} & {\tt \%io\_buffer} & integer \\
  {\tt more-toraldo-search-length} & {\tt \%more\_toraldo} & integer \\
  {\tt history-length-for-non-monotone-descent} & {\tt \%non\_monotone} & integer \\
  {\tt first-derivative-approximations} & {\tt \%first\_derivatives} & string \\
  {\tt second-derivative-approximations} & {\tt \%second\_derivatives} & string \\
  {\tt primal-accuracy-required} & {\tt \%stopc} & real \\
  {\tt dual-accuracy-required} & {\tt \%stopg} & real \\
  {\tt minimum-merit-value} & {\tt \%min\_aug} & real \\
  {\tt inner-iteration-relative-accuracy-required} & {\tt \%acccg} & real \\
  {\tt initial-trust-region-radius} & {\tt \%initial\_radius} & real \\
  {\tt maximum-radius} & {\tt \%maximum\_radius} & real \\
  {\tt eta-successful} & {\tt \%eta\_successful} & real \\
  {\tt eta-very-successful} & {\tt \%eta\_very\_successful} & real \\
  {\tt eta-extremely-successful} & {\tt \%eta\_extremely\_successful} & real \\
  {\tt gamma-smallest} & {\tt \%gamma\_smallest} & real \\
  {\tt gamma-decrease} & {\tt \%gamma\_decrease} & real \\
  {\tt gamma-increase} & {\tt \%gamma\_increase} & real \\
  {\tt mu-meaningful-model} & {\tt \%mu\_meaningful\_model} & real \\
  {\tt mu-meaningful-group} & {\tt \%mu\_meaningful\_group} & real \\
  {\tt initial-penalty-parameter} & {\tt \%initial\_mu} & real \\
  {\tt no-dual-updates-until-penalty-parameter-below} & {\tt \%mu\_tol} & real \\
  {\tt initial-primal-accuracy-required} & {\tt \%firstc} & real \\
  {\tt initial-dual-accuracy-required} & {\tt \%firstg} & real \\
  {\tt pivot-tolerance-used} & {\tt \%SILS\_cntl\%u} & real \\
  {\tt maximum-cpu-time-limit} & {\tt cpu\_time\_limit} & real \\
  {\tt quadratic-problem} & {\tt \%quadratic\_problem} & logical \\
  {\tt two-norm-trust-region-used} & {\tt \%two\_norm\_tr} & logical \\
  {\tt exact-GCP-used} & {\tt \%exact\_gcp} & logical \\
  {\tt magical-steps-allowed} & {\tt \%magical\_steps} & logical \\
  {\tt subproblem-solved-accurately} & {\tt \%accurate\_bqp} & logical \\
  {\tt structured-trust-region-used} & {\tt \%structured\_tr} & logical \\
  {\tt print-for-maximization} & {\tt \%print\_max} & logical \\
  {\tt print-full-solution} & {\tt \%full\_solution} & logical \\
  {\tt alive-filename} & {\tt \%alive\_file} & character \\ \hline

\ectable{\label{specfile}Specfile commands and associated
components of {\tt control}. See Table~\ref{string} for
permitted string values.}

\bctable{|l|l|l|}
\hline
  command & string value & {\tt control} component assignment \\
\hline
   & {\tt CG} & {\tt \%linear\_solver = 1} \\
   & {\tt DIAGONAL\_CG} & {\tt \%linear\_solver = 2} \\
   & {\tt USERS\_CG} & {\tt \%linear\_solver = 3} \\
   & {\tt EXPANDING\_BAND\_CG} & {\tt \%linear\_solver = 4} \\
   & {\tt MUNKSGAARD\_CG} & {\tt \%linear\_solver = 5} \\
  {\tt linear-solver-used}
   & {\tt SCHNABEL\_ESKOW\_CG} & {\tt \%linear\_solver = 6} \\
   & {\tt GMPS\_CG} & {\tt \%linear\_solver = 7} \\
   & {\tt BAND\_CG} & {\tt \%linear\_solver = 8} \\
   & {\tt LIN\_MORE\_CG} & {\tt \%linear\_solver = 9} \\
   & {\tt MULTIFRONTAL} & {\tt \%linear\_solver = 11} \\
   & {\tt MODIFIED\_MULTIFRONTAL} & {\tt \%linear\_solver = 12} \\
\hline
   & {\tt EXACT} & {\tt \%first\_derivatives = 0} \\
  {\tt first-derivative-approximations}
   & {\tt FORWARD} & {\tt \%first\_derivatives = 1} \\
   & {\tt CENTRAL} & {\tt \%first\_derivatives = 2} \\
\hline
   & {\tt EXACT} & {\tt \%second\_derivatives = 0} \\
   & {\tt BFGS} & {\tt \%second\_derivatives = 1} \\
  {\tt second-derivative-approximations}
   & {\tt DFP} & {\tt \%second\_derivatives = 2} \\
   & {\tt PSB} & {\tt \%second\_derivatives = 3} \\
   & {\tt SR1} & {\tt \%second\_derivatives = 4} \\
\hline
\ectable{\label{string}String values and their interpretation.}

\itt{device} is a scalar \intentin argument of type \integer,
that must be set to the unit number on which the specfile
has been opened. If {\tt device} is not open, {\tt control} will
not be altered and execution will continue, but an error message
will be printed on unit {\tt control\%error}.

\end{description}

%%%%%%%%%%%%%%%%%%%%%% Information printed %%%%%%%%%%%%%%%%%%%%%%

\galinfo
The user is able to control the amount of intermediate printing
performed in the course of the minimization. Printing is under
the control of the parameter {\tt control\%print\_level} and output is sent to
I/O unit number {\tt control\%error}.
Possible values of {\tt control\%print\_level} and the levels of output
produced are as follows.
\begin{description}
\itt{control\%print\_level $\leq 0.$} No printing, except warning messages,
   will be performed.
\itt{control\%print\_level $\geq 1.$} Details of the minimization function
                 will be output.
                  This includes the number of variables,
                  groups and
                  nonlinear elements which are used and a list of the
                  variables which occur in each
                  of the linear and nonlinear elements in every group.

                 If the current iterate provides an acceptable
                 estimate of the minimizer of the augmented Lagrangian
                 function, the two-norm of the general constraints and
                 the current value of the penalty parameter are given.
\itt{control\%print\_level $= 1.$} A simple one line description of each
                 iteration is
                 given. This includes the iteration number, the
                 number of derivative evaluations that have been
                 made, the number of conjugate-gradient iterations
                 that have been performed,
                 the current value of the augmented Lagrangian
                 function, the (two-) norm
                 of the projected gradient,
                 the ratio $\rho$ of the actual to predicted decrease
                 in augmented Lagrangian function value achieved,
                 the current trust-region radius,
                 the norm of the step taken,
                 an indication of how the direct or iterative
                 method ended, the number of variables which lie
                 away from their bounds and the total time
                 spent on the minimization.
\itt{control\%print\_level $= 2.$} In addition to the information output with
                 {\tt control\%print\_level = 1},
                 a short description of the approximate solution
                 to the inner-iteration linear system is given.
                 Before a successful ({\tt inform\%status = 0}) exit, details of
                 the estimate of the minimizer and the
                 gradient of the augmented Lagrangian
                 function are given.
\itt{control\%print\_level $= 3.$} A list of the current iteration number,
                 the value of the augmented Lagrangian function, the
                 number of derivative evaluations that have been
                 made, the (two-) norm
                 of the projected gradient,
                 the number of conjugate gradients
                 iterations that have been performed and the
                 current trust-region radius
                 are given, followed by the
                 current estimate of the minimizer.
                 The values of the reduction in the model of the
                 augmented Lagrangian function
                 and the actual reduction in this
                 function, together with their ratio, are also given.
                 Before a successful ({\tt inform\%status = 0}) exit, details of
                 the estimate of the minimizer and the
                 gradient of the augmented Lagrangian function are given.

                 If the current iterate also provides an acceptable
                 estimate of the minimizer of the augmented Lagrangian
                 function, values of the general constraints and
                 estimates of the Lagrange multipliers are also given.
\itt{control\%print\_level $= 4.$} In addition to the information output with
                 {\tt control\%print\_level= 3},
                 the gradient of the augmented Lagrangian function at
                 the current estimate of the minimizer is given.
                 Full details of the approximate solution
                 to the inner-iteration linear system are also given.
                 This level of output is intended as a debugging aid
                 for the expert only.
\itt{control\%print\_level $= 5.$} In addition to the information output with
                 {\tt control\%print\_level = 4},
                 the diagonal elements of the second derivative
                 approximation are given.
\itt{control\%print\_level $\geq 6.$} In addition to the information output with
                  {\tt control\%print\_level = 5},
                  the second derivative approximations
                  (taken with respect to the internal variables)
                  to each nonlinear element function are given.
\end{description}

%%%%%%%%%%%%%%%%%%%%%% GENERAL INFORMATION %%%%%%%%%%%%%%%%%%%%%%

\galgeneral

\galcommon None.
\galworkspace Provided automatically by the module.
\galroutines The user must provide an external {\tt RANGE} subroutine
(see Section~\ref{internal_and_ranges}), and optionally may provide
external {\tt ELFUN} (see Section~\ref{efv}),
{\tt ELFUN\_flexible} (see Section~\ref{efvf}) and
{\tt GROUP} (see Section~\ref{gfv}) subroutines. Since these
routines are all dummy arguments to \solver\, their names may be
adapted for the user's purposes.
\galmodules
{\tt GALAHAD\_CPU\_time},
{\tt GALAHAD\_SPECFILE},
{\tt LANCELOT\_INITW},
{\tt LANCELOT\_OTHERS},
{\tt LAN\-CELOT\_HSPRD},
{\tt LANCELOT\_CAUCHY},
{\tt LANCELOT\_CG},
{\tt LANCELOT\_PRECN},
{\tt LANCELOT\_FRNTL},
{\tt LANCELOT\_STRUTR},
{\tt GAL\-AHAD\_SMT},
{\tt GAL\-AHAD\_SILS},
{\tt GALAHAD\_SCU},
{\tt LANCELOT\_ASMBL},
{\tt LANCELOT\_EXTEND}.
\galio No input; output on device numbers {\tt control\%out} and
{\tt control\%error}.
    Output is provided under the control of {\tt control\%print\_level}.
     If the user supplies positive unit numbers to {\tt control\%out} and
{\tt control\%error}, messages
     are printed to the user supplied units. However if one of unit
     numbers is non-positive, printing to this unit is suppressed.
\galrestrictions
     {\tt prob\%n $>$ 0},
     {\tt prob\%ng $>$ 0},
     {\tt prob\%nel $\geq$ 0}.
\galportability ISO Fortran~95 + TR 15581 or Fortran~2003. The package is thread-safe.

%%%%%%%%%%%%%%%%%%%%%% METHOD %%%%%%%%%%%%%%%%%%%%%%

\galmethod
The basic method implemented within {\tt \packagename} is described in
detail by Conn, Gould and Toint (1991).  The method used to solve the
inner iteration subproblem is described by Conn, Gould and Toint
(1988b). The concept of partial separability was first suggested by
Griewank and Toint (1982).  The extension to group partially separable
functions was given by Conn, Gould and Toint (1990).  Also see Chapter~3
of the {\sf LANCELOT A} manual.

The Lagrangian function associated with objective function
\req{objective} and general constraints \req{constraints} is the
composite function \disp{\ell( \bmx, \bmy ) = f(\bmx) + \sum_{i \in
\calG_C} y_i c_i (\bmx).}  The scalars $y_i$ are known as Lagrange
multiplier estimates.  At a solution $\bmx^{\star}$ to the constrained
minimization problem, there are Lagrange multipliers $\bmy^{\star}$ for
which the components of the gradient of the Lagrangian function
$\partial \ell(\bmx^{\star}, \bmy^{\star},) / \partial x_i$ $= 0$ whenever
the corresponding variable $x_i^{\star}$ lies strictly between its lower
and upper bounds.  It is useful, but not essential, for the initial
Lagrange multiplier estimates to be close to $\bmy^{\star}$.

The augmented Lagrangian function is the composite function
\eqn{objectivec}{\phi( \bmx, \bmy, \mu ) = \ell( \bmx, \bmy) +
 \frac{1}{2 \mu} \sum_{i \in \calG_C} (c_i (\bmx))^2,}
where $\mu$ is known as the penalty parameter.  An inner iteration is
used to find an approximate minimizer of (\ref{objectivec}) within the
feasible box for fixed values of the penalty parameter and Lagrange
multiplier estimates. The outer iteration of {\tt \packagename}
automatically adjusts the penalty parameter and Lagrange multiplier
estimates to ensure convergence of these approximate minimizers to a
solution of the constrained optimization problem. Each inner iteration
is terminated as soon as a norm of the projected gradient,
\disp{\min(\max( \bmx - \nabla_x  \phi( \bmx, \bmy, \mu ), \bmx^l), \bmx^u) 
 - \bmx,}
is sufficiently small.

In the inner iteration, a step from the current estimate of the solution
is determined using a trust-region approach.  That is, a quadratic model
of the augmented Lagrangian function is approximately minimized within
the intersection of the constraint ``box'' and another convex region,
the trust-region.  This minimization is carried out in two stages.
Firstly, the so-called generalized Cauchy point for the quadratic
subproblem is found.  (The purpose of this point is to ensure that the
algorithm converges and that the set of bounds which are satisfied as
equations at the solution is rapidly identified.)  Thereafter an
improvement to the quadratic model is sought using either a
direct-matrix or truncated conjugate-gradient algorithm. The
trust-region size is increased if the reduction obtained in the
objective function is reasonable when compared with the reduction
predicted by the model and reduced otherwise.

A central idea is that a collection of small matrices approximating the
Hessian matrices of each $e_k$ is used and updated at every iteration
using one of a number of possible updating formulae.  Augmented
Lagrangian function values and derivatives are assembled from these
components as required.

The strategy for treating bound constraints is based on the usual
projection device and is described in detail in Conn, Gould and Toint (1988a).
Of the new features in LANCELOT B,
structured trust regions are described in
Conn, Gould and Toint (2000), Section 10.2, and
non-monotone descent, {\em ibid}, Section 10.1,
while the alternative projection is due to Mor\'{e} and Toraldo (1991),
and the alternative incomplete Cholesky factorization is
described by Lin and Mor\'{e} (1998)

\galreferences
\vspace*{1mm}

\noindent
The basic method is described in detail in
\vspace*{1mm}

\noindent
%\begin{center}
A. R. Conn, N. I. M. Gould and Ph. L. Toint (1992).
LANCELOT. A fortran package for large-scale nonlinear optimization
(release A). Springer Verlag Series in Computational Mathematics 17,
Berlin,
%\end{center}

\noindent
and details of its computational performance may be found in

\noindent
%\begin{center}
A. R. Conn, N. I. M. Gould and Ph. L. Toint (1996).
Numerical experiments with the {\sf LANCELOT} package
(Release A) for large-scale nonlinear optimization
Mathematical Programming {\bf 73} 73-110.
%\end{center}

%\noindent
Convergence properties of the method are described in

\noindent
%\begin{center}
A. R. Conn, N. I. M. Gould and Ph. L. Toint (1991).
A Globally Convergent Augmented {L}agrangian Algorithm for
Optimization with General Constraints and Simple Bounds.
SIAM Journal on Numerical Analysis {\bf 28} 545-572,
%\end{center}

\noindent
and

\noindent
%\begin{center}
A. R. Conn, N. I. M. Gould and Ph. L. Toint (1988a).
Global convergence of a class of trust region algorithms
for optimization with simple bounds.
SIAM Journal on Numerical Analysis {\bf 25} 433-460,
%\end{center}

\noindent
while details of the inner iteration are provided by

\noindent
%\begin{center}
A. R. Conn, N. I. M. Gould and Ph. L. Toint (1988b).
Testing a class of methods for solving minimization
problems with simple bounds on the variables.
Mathematics of Computation {\bf 50} 399-430.
%\end{center}

%\noindent
Partial separability was introduced by

\noindent
%\begin{center}
A. Griewank and Ph. L. Toint (1982).
Partitioned variable metric updates for large structured
optimization problems.
Numerische Mathematik {\bf 39} 119-137,
%\end{center}

\noindent
and its generalization to group partial separability was given by

\noindent
%\begin{center}
A. R. Conn, N. I. M. Gould and Ph. L. Toint (1990).
An introduction to the structure of large scale nonlinear
optimization problems and the {{\sf LANCELOT}} project.
In ``Computing Methods in Applied Sciences and Engineering''
(R. Glowinski and A. Lichnewsky, eds), SIAM, Philadelphia, 42-51.
%\end{center}

%\noindent
Many of the newer issues are discussed by

\noindent
%\begin{center}
A. R. Conn, N. I. M. Gould and Ph. L. Toint (2000).
Trust Region Methods.
SIAM, Philadelphia,
%\end{center}

\noindent
while the alternative incomplete Cholesky factorization is
due to

\noindent
%\begin{center}
C.-J. Lin and J. J. Mor\'{e} (1999).
Incomplete Cholesky factorizations with limited memory.
SIAM Journal on Scientific Computing {\bf 21} 21-45,
%\end{center}

\noindent
and the alternative projection originally described in

\noindent
%\begin{center}
J. J. Mor\'{e} and G. Toraldo (1991)
On the solution of large quadratic programming problems
with bound constraints.
SIAM Journal on Optimization {\bf 1} 93-113,
%\end{center}

\noindent
and given in the context used here by

\noindent
%\begin{center}
C.-J. Lin and J. J. Mor\'{e} (1999b).
Newton's method for large bound-constrained optimization problems,
SIAM Journal on Optimization {\bf 9} 1100-1127.
%\end{center}

%%%%%%%%%%%%%%%%%%%%%% EXAMPLE %%%%%%%%%%%%%%%%%%%%%%

\galexample
We now consider the small example problem,
\disp{{\rm minimize} \;\; f(x_1^{ },x_2^{ },x_3^{ }) =
 x_1^2 + x_2^{ } \sin(x_1^{ } + x_3^{ }) + 3 x_2^4 x_3^4 + x_2^{ }
+ 2 x_1^2 x_2^2 }
subject to the general constraint
\disp{c(x_1,x_2,x_3) = \cos(x_1 + 2 x_2 - 1 ) = 0}
and the bounds $- 1 \leq x_2 \leq 1$ and
$1 \leq x_3 \leq 2$.
There are a number of ways of casting this problem in the
framework required by \solver.
Here, we consider partitioning $f$
into groups as
\begin{center}
\begin{tabular}{ccccccccc}
$(x_1)^2$  &+& $(x_2 \sin(x_1+x_3))$ &+& $3(x_2 x_3)^4$ &+& $(x_2)$   &+& $2(x_1 x_2)^2$  \\
$\uparrow$ & &     $\uparrow$        & & $\uparrow$     & & $\uparrow$ & & $\uparrow$\\
group 1    & &       group 2         & &   group 3      & & group 4    & & group 5
\end{tabular}
\end{center}

Similarly, we write $c$ as
\begin{center}
\begin{tabular}{cc}
$\cos(x_1 + 2 x_2 - 1)$. & \\ $\uparrow$ & \\ group 6 &
\end{tabular}
\end{center}

Notice the following:
\begin{enumerate}
\item
Group 1 uses the non-trivial
   group function $g_1(\alpha) = \alpha^2$.
   The group contains a single {\em linear} element;
   the element function is $x_1$.

\item
Group 2 uses the trivial
   group function $g_2(\alpha) = \alpha$.
   The group contains a single {\em nonlinear} element;
   this element function is $e_1(x_1,x_2,x_3) = x_2 \sin( x_1 + x_3 )$.
   The element function has {\em three} elemental variables,
   $v_1$, $v_2$ and $v_3$, say, (with $v_1 = x_2$,
   $v_2 = x_1$ and $v_3 = x_3$), but may
   be expressed in terms of {\em two} internal variables
   $u_1$ and $u_2$, say, where
   $u_1 = v_1$ and $u_2 = v_2 + v_3$.

\item
Group 3 uses the non-trivial
   group function $g_3( \alpha ) = \alpha^4$, weighted by the
   factor 3.
   Again, the group contains a single {\em nonlinear} element;
   this element function is $e_2(x_2,x_3) = x_2 x_3$.
   The element function has {\em two} elemental variables,
   $v_1$ and $v_2$, say, (with $v_1 = x_2$ and
   $v_2 = x_3$). This time, however, there is no useful
   transformation to internal variables.

\item
Group 4 again uses the trivial
   group function $g_4(\alpha ) = \alpha$.
   This time the group contains a single {\em linear} element
   $x_2$.

\item
Group 5 uses the
   group function $g_5(\alpha ) = \alpha^2$, weighted by the factor 2.
   Notice that this is the same {\em type} as occurred in group 1.
   The group contains a single nonlinear element, $e_3(x_1,x_2) = x_1x_2$.
   But notice that this element is of the same {\em type} (i.e., $v_1 v_2$
   for appropriate $v_1$ and $v_2$) as $e_2$ featured in group 3
   although it uses different elemental variables (this time
   $v_1 = x_1$ and $v_2 = x_2$).
\item
Finally, group 6 uses the non-trivial
   group function $g_6(\alpha) = \cos(\alpha)$.
   The group contains a single {linear} element;
   the element function is $x_1 + 2 x_2 - 1$.
\end{enumerate}

Thus we see that we can consider our objective
and constraint
functions to be made up of six group functions;
the first, third, fifth and sixth are non-trivial
so we need to provide function and derivative values for these when
prompted by \solver. Of the four non-trivial groups, there are only
three different types, since groups one and five are structurally the same.
There are three nonlinear elements, one each from groups two, three and five.
Again this means that we need to provide function and derivative
values for these when required by \solver, but there is a slight
simplification as elements two and three are of the same type.
Finally, one of these elements, the first,
has a useful transformation from elemental to internal variables
so the routine {\tt RANGE} must be set to provide this transformation.

The problem involves three variables, six groups
and three nonlinear elements so we set
      {\tt prob\%n} $= 3$,
      {\tt prob\%ng} $= 6$ and
      {\tt prob\%nel} $= 3$.
As the first, third, fifth and sixth group functions
are non-trivial, we set
{\tt prob\%GXEQX} as:

\begin{center}
{\tt \begin{tabular}{ccccccc}
i        & 1 & 2 & 3 & 4 & 5 & 6\\
prob\%GXEQX(i) & .FALSE. & .TRUE. & .FALSE. & .TRUE. & .FALSE. & .FALSE.
\end{tabular}}
\end{center}
Non-trivial groups 1 and 5 involve functions of the same type
(we label this a ``type 1'' group; the actual label is
unimportant, so long as groups of different types have different labels),
and these are different from
those for groups 2 and 6 (which we call ``types 2'' and ``3'')---by
convention, trivial groups are of ``type 0'' . So we
set {\tt prob\%ITYPEG} as:

\begin{center}
{\tt \begin{tabular}{ccccccc}
i        & 1 & 2 & 3 & 4 & 5 & 6\\
prob\%ITYPEG(i) & 1 & 0 & 2 & 0 & 1 & 3
\end{tabular}}
\end{center}
The first five groups are associated with the objective function
while the sixth defines a constraint so we set {\tt prob\%KNDOFG} as:

\begin{center}
{\tt \begin{tabular}{ccccccc}
i        & 1 & 2 & 3 & 4 & 5 & 6\\
prob\%KNDOFG(i) & 1 & 1 & 1 & 1 & 1 & 2
\end{tabular}}
\end{center}

The first nonlinear element occurs in group 2, the second
in group 3 and the last in group 5; each element is unweighted.
We thus set {\tt prob\%IELING}, {\tt prob\%ESCALE} and {\tt prob\%ISTADG} as:

\begin{center}
{\tt \begin{tabular}{cccc}
i         & 1 & 2 & 3\\
prob\%IELING(i) & 1 & 2 & 3\\
prob\%ESCALE(i) & 1.0 & 1.0 & 1.0
\end{tabular}}
\end{center}

\begin{center}
{\tt \begin{tabular}{cccccccc}
i         & 1 & 2 & 3 & 4 & 5 & 6 & 7\\
prob\%ISTADG(i) & 1 & 1 & 2 & 3 & 3 & 4 & 4
\end{tabular} }
\end{center}
The first nonlinear element function is not a quadratic function
of its internal variables so we set {\tt control\%quadratic\-\_problem=.FALSE.}.
The second and third nonlinear element functions are of the
same type (we number this a ``type 2'' element), while
the first nonlinear element is different (a ``type 1'' element).
So we set {\tt prob\%ITYPEE} as:

\begin{center}
{\tt \begin{tabular}{cccc}
i        & 1 & 2 & 3 \\
prob\%ITYPEE(i) & 1 & 2 & 2
\end{tabular}}
\end{center}
Nonlinear element one, $e_1$ assigns variables $x_2$, $x_1$ and
$x_3$ to its elemental variables, while
the second and third nonlinear elements $e_2$ and $e_3$
assign variables $x_2$ and
$x_3$, and $x_1$ and $x_2$ to their elemental variables, respectively.
Hence {\tt prob\%IELVAR} contains:

\begin{center}
{\tt \begin{tabular}{clcrlrlr}
i         & ~1 & ~~2 & ~~3~ & ~~4 & 5~ & ~~6 & 7~ \\
prob\%IELVAR(i) & ~2 & ~~1 & ~~3~ & ~~2 & 3~ & ~~1 & 2~ \\
&
\multicolumn{3}{c}{$\longleftarrow$ {$e_1$} $\longrightarrow$} &
\multicolumn{2}{c}{$\leftarrow$ {$e_2$} $\rightarrow$} &
\multicolumn{2}{c}{$\leftarrow$ {$e_3$} $\rightarrow$}
\end{tabular} }
\end{center}
In this vector, we now locate the position of the first
variable of each nonlinear element, and build the vector {\tt prob\%ISTAEV} as
follows:

\begin{center}
{\tt \begin{tabular}{ccccc}
i         & 1 & 2 & 3 & 4\\
prob\%ISTAEV(i) & 1 & 4 & 6 & 8
\end{tabular} }
\end{center}
All three nonlinear elements have two internal variables --- elements
two and three using their elemental variables as internals --- so we set
{\tt prob\%INTVAR} as follows:

\begin{center}
{\tt \begin{tabular}{ccccc}
i         & 1 & 2 & 3 & 4\\
prob\%INTVAR(i) & 2 & 2 & 2 & -
\end{tabular} }
\end{center}
As the first nonlinear element has a useful transformation
between elemental and internal variables, while the other two do not,
{\tt prob\%INTREP} contains:

\begin{center}
{\tt \begin{tabular}{cccc}
i         & 1 & 2 & 3 \\
prob\%INTREP(i) & .TRUE. & .FALSE. & .FALSE.
\end{tabular} }
\end{center}

Turning to the linear elements,
groups one, four and six each have
a linear element.
Those from groups one and four
involve a single variable --- $x_1$ for
group one and $x_2$ for group four. That from group six involves
the variables $x_1$ and $x_3$. All the coefficients, excepting the last,
are one; the last is two.
We thus set {\tt prob\%ISTADA} as:

\begin{center}
{\tt \begin{tabular}{cccccccc}
i         & 1 & 2 & 3 & 4 & 5 & 6 & 7\\
prob\%ISTADA(i) & 1 & 2 & 2 & 2 & 3 & 3 & 5
\end{tabular} }
\end{center}
and {\tt prob\%ICNA} and {\tt prob\%A} as

\begin{center}
{\tt \begin{tabular}{ccccc}
i         & 1 & 2 & 3 & 4 \\
prob\%ICNA(i)   & 1 & 2 & 1 & 2 \\
prob\%A(i)      & 1.0D+0 & 1.0D+0 & 1.0D+0 & 2.0D+0
\end{tabular} }
\end{center}
Only the last group
has a constant term so {\tt prob\%B} is set to:

\begin{center}
{\tt \begin{tabular}{ccccccc}
i         & 1 & 2 & 3 & 4 & 5 & 6 \\
prob\%B(i)      & 0.0D+0 & 0.0D+0  & 0.0D+0 & 0.0D+0 & 0.0D+0 & 1.0D+0
\end{tabular} }
\end{center}

The bounds
for the problem are set in {\tt prob\%BL} and {\tt prob\%BU}. As the first
variable is allowed to take any value, we specify lower and
upper bounds of $\pm 10^{20}$. Thus we set

\begin{center}
{\tt \begin{tabular}{cccc}
i         & 1 & 2 & 3 \\
prob\%BL(i)     & -1.0D+20 & -1.0D+0  & -1.0D+0 \\
prob\%BU(i)     & ~1.0D+20 & ~1.0D+0  & ~2.0D+0
\end{tabular} }
\end{center}

The 3rd group is scaled by 3.0, the fifth by 2.0,  while the others
are  unscaled. Thus we set {\tt prob\%GSCALE} to:

\begin{center}
{\tt \begin{tabular}{ccccccc}
i         & 1 & 2 & 3 & 4 & 5 & 6 \\
prob\%GSCALE(i) & 1.0D+0 & 1.0D+0  & 3.0D+0 & 1.0D+0 & 2.0D+0 & 1.0D+0
\end{tabular} }
\end{center}
It is unclear that the variables are badly scaled so we
set {\tt prob\%VSCALE} to:

\begin{center}
{\tt \begin{tabular}{cccc}
i         & 1 & 2 & 3 \\
prob\%VSCALE(i) & 1.0D+0 & 1.0D+0 & 1.0D+0
\end{tabular} }
\end{center}

Finally, we choose to allocate the names {\tt Obj 1} to {\tt Obj 5} to
the four groups
associated with the objective function
and call the constraint
group {\tt Constraint}. The variables are called {\tt x 1}
to {\tt x 3}. We thus set {\tt prob\%GNAMES} and {\tt prob\%VNAMES} to:

\begin{center}
{\tt \begin{tabular}{ccccccc}
i         & 1 & 2 & 3 & 4 & 5 & 6 \\
prob\%GNAMES(i) & Obj 1 & Obj 2  & Obj 3 & Obj 4 & Obj 5 & Constraint
\end{tabular} }
\end{center}
and

\begin{center}
{\tt \begin{tabular}{cccc}
i         & 1 & 2 & 3 \\
prob\%VNAMES(i) & x 1 & x 2 & x 3
\end{tabular} }
\end{center}

Notice that, in our example, neither the group nor nonlinear element functions
depends upon parameters, so both {\tt prob\%EPVALU} and {\tt prob\%GPVALU} are
empty, and consequently all components of the parameter starting address arrays
{\tt prob\%ISTEPA} and {\tt prob\%ISTGPA} should be set to {\tt 0}.

\solver\ leaves the computation of function and derivative values of the
group and nonlinear element functions to the user---this may either be
done internally (by providing suitable {\tt GROUP} and {\tt ELFUN}
subroutines, or externally via reverse communication.  We need only to
specify the group functions for the non-trivial groups, groups 1, 3 and
5.  Also note that the function and gradient values are only evaluated
if the component is specified in {\tt ICALCG}.  A suitable {\tt GROUP}
subroutine might be the following:

\def\baselinestretch{0.8}
{%\footnotesize
{\tt \begin{verbatim}
   SUBROUTINE GROUP ( GVALUE, lgvalu, FVALUE, GPVALU, ncalcg, &
                      ITYPEG, ISTGPA, ICALCG, ltypeg, lstgpa, &
                      lcalcg, lfvalu, lgpvlu, derivs, igstat )
   INTEGER, PARAMETER :: wp = KIND( 1.0D+0 ) ! set precision
   INTEGER, INTENT( IN ) :: lgvalu, ncalcg
   INTEGER, INTENT( IN ) :: ltypeg, lstgpa, lcalcg, lfvalu, lgpvlu
   INTEGER, INTENT( OUT ) :: igstat
   LOGICAL, INTENT( IN ) :: derivs
   INTEGER, INTENT( IN ), DIMENSION ( ltypeg ) :: ITYPEG
   INTEGER, INTENT( IN ), DIMENSION ( lstgpa ) :: ISTGPA
   INTEGER, INTENT( IN ), DIMENSION ( lcalcg ) :: ICALCG
   REAL ( KIND = wp ), INTENT( IN ), DIMENSION ( lfvalu ) :: FVALUE
   REAL ( KIND = wp ), INTENT( IN ), DIMENSION ( lgpvlu ) :: GPVALU
   REAL ( KIND = wp ), INTENT( OUT ), DIMENSION ( lgvalu, 3 ) :: GVALUE
   INTEGER :: igrtyp, igroup, ipstrt, jcalcg
   REAL ( KIND = wp ) :: alpha, alpha2
   igstat = 0
   DO jcalcg = 1, ncalcg
     igroup = ICALCG( jcalcg )
     igrtyp = ITYPEG( igroup )
     IF ( igrtyp == 0 ) CYCLE ! skip if the group is trivial
     ipstrt = ISTGPA( igroup ) - 1
     SELECT CASE ( igrtyp )
     CASE ( 1 )
       alpha = FVALUE( igroup )
       IF ( .NOT. derivs ) THEN
         GVALUE( igroup, 1 ) = alpha * alpha
       ELSE
         GVALUE( igroup, 2 ) = 2.0_wp * alpha
         GVALUE( igroup, 3 ) = 2.0_wp
       END IF
     CASE ( 2 )
       alpha  = FVALUE( igroup )
       alpha2 = alpha * alpha
       IF ( .NOT. derivs ) THEN
         GVALUE( igroup, 1 ) = alpha2 * alpha2
       ELSE
         GVALUE( igroup, 2 ) = 4.0_wp * alpha2 * alpha
         GVALUE( igroup, 3 ) = 12.0_wp * alpha2
       END IF
     CASE ( 3 )
       alpha  = FVALUE( igroup )
       IF ( .NOT. derivs ) THEN
         GVALUE( igroup, 1 ) = COS( alpha )
       ELSE
         GVALUE( igroup, 2 ) = - SIN( alpha )
         GVALUE( igroup, 3 ) = - COS( alpha )
       END IF
     END SELECT
   END DO
   RETURN
   END SUBROUTINE GROUP
\end{verbatim} } }
\def\baselinestretch{1.0}

\noindent
Here, the logical parameter {\tt derivs} set \true\ specifies that the first
and second derivatives are required; a \false\ value will return the
function values.

To evaluate the values and derivatives of the nonlinear element
functions, the following {\tt ELFUN} subroutine would be suitable:

\def\baselinestretch{0.8}
{%\footnotesize
{\tt \begin{verbatim}
   SUBROUTINE ELFUN ( FUVALS, XVALUE, EPVALU, ncalcf, ITYPEE, ISTAEV, &
                      IELVAR, INTVAR, ISTADH, ISTEPA, ICALCF, ltypee, &
                      lstaev, lelvar, lntvar, lstadh, lstepa, lcalcf, &
                      lfuval, lxvalu, lepvlu, ifflag, ifstat )
   INTEGER, PARAMETER :: wp = KIND( 1.0D+0 ) ! set precision
   INTEGER, INTENT( IN ) :: ncalcf, ifflag, ltypee, lstaev, lelvar, lntvar
   INTEGER, INTENT( IN ) :: lstadh, lstepa, lcalcf, lfuval, lxvalu, lepvlu
   INTEGER, INTENT( OUT ) :: ifstat
   INTEGER, INTENT( IN ) :: ITYPEE( LTYPEE ), ISTAEV( LSTAEV ), IELVAR( LELVAR )
   INTEGER, INTENT( IN ) :: INTVAR( LNTVAR ), ISTADH( LSTADH ), ISTEPA( LSTEPA )
   INTEGER, INTENT( IN ) :: ICALCF( LCALCF )
   REAL ( KIND = wp ), INTENT( IN ) :: XVALUE( LXVALU )
   REAL ( KIND = wp ), INTENT( IN ) :: EPVALU( LEPVLU )
   REAL ( KIND = wp ), INTENT( INOUT ) :: FUVALS( LFUVAL )
   INTEGER :: ielemn, ieltyp, ihstrt, ilstrt, igstrt, ipstrt, jcalcf
   REAL ( KIND = wp ) :: v1, v2, u1, u2, u3, cs, sn
   ifstat = 0
   DO jcalcf = 1, ncalcf
     ielemn = ICALCF( jcalcf )
     ilstrt = ISTAEV( ielemn ) - 1
     igstrt = INTVAR( ielemn ) - 1
     ipstrt = ISTEPA( ielemn ) - 1
     IF ( ifflag == 3 ) ihstrt = ISTADH( ielemn ) - 1
     ieltyp = ITYPEE( ielemn )
     SELECT CASE ( ieltyp )
     CASE ( 1 )
       u1 = XVALUE( IELVAR( ilstrt + 1 ) )
       u2 = XVALUE( IELVAR( ilstrt + 2 ) )
       u3 = XVALUE( IELVAR( ilstrt + 3 ) )
       v1 = u1
       v2 = u2 + u3
       cs = COS( v2 )
       sn = SIN( v2 )
       IF ( ifflag == 1 ) THEN
         FUVALS( ielemn ) = v1 * sn
       ELSE
         FUVALS( igstrt + 1 ) = sn
         FUVALS( igstrt + 2 ) = v1 * cs
         IF ( ifflag == 3 ) THEN
           FUVALS( ihstrt + 1 ) = 0.0_wp
           FUVALS( ihstrt + 2 ) = cs
           FUVALS( ihstrt + 3 ) = - v1 * sn
         END IF
       END IF
     CASE ( 2 )
       u1 = XVALUE( IELVAR( ilstrt + 1 ) )
       u2 = XVALUE( IELVAR( ilstrt + 2 ) )
       IF ( ifflag == 1 ) THEN
         FUVALS( ielemn ) = u1 * u2
       ELSE
         FUVALS( igstrt + 1 ) = u2
         FUVALS( igstrt + 2 ) = u1
         IF ( ifflag == 3 ) THEN
           FUVALS( ihstrt + 1 ) = 0.0_wp
           FUVALS( ihstrt + 2 ) = 1.0_wp
           FUVALS( ihstrt + 3 ) = 0.0_wp
         END IF
       END IF
     END SELECT
   END DO
   RETURN
   END SUBROUTINE ELFUN
\end{verbatim} } }
\def\baselinestretch{1.0}

The integer parameter ifflag specifies whether function values
({\tt ifflag = 1}) or derivatives ({\tt ifflag > 1}) are required.
When {\tt ifflag = 2}, just first derivatives are required, while
{\tt ifflag = 3} also requests second derivatives.
Notice that the derivatives are taken with respect to
the internal variables.
For the first nonlinear element function,
this means that we must transform from elemental to internal
variables before evaluating the derivatives. Thus, as
this function may be written as $f(v_1,v_2)=v_1 \sin v_2$,
where $v_1 = u_1$ and $v_2 = u_2 + u_3$,
the gradient and Hessian matrix are
\disp{\vect{\sin v_2 \\ v_1 \cos v_2} \; \; \; \; {\rm and} \; \; \; \;
 \mat{cc}{ 0 & \cos v_2 \\ \cos v_2 & -v_1 \sin v_2},}
respectively. Notice that it is easy to specify the second
derivatives
for this example so we do so.
Also note that the function and gradient values
are only evaluated if the component is specified in {\tt ICALCF}.

We must also specify the routine {\tt RANGE} for our example.  As we
have observed, only the first element has a useful transformation.  The
transformation matrix is
\disp{\mat{ccc}{1 & 0 & 0 \\ 0 & 1 & 1}}
(see Section~\ref{internal_and_ranges}).
As a consequence, the following routine {\tt RANGE} is appropriate:

\def\baselinestretch{0.8}
{%\footnotesize
{\tt \begin{verbatim}
   SUBROUTINE RANGE( ielemn, transp, W1, W2, nelvar, ninvar, ieltyp, lw1, lw2 )
   INTEGER, PARAMETER :: wp = KIND( 1.0D+0 ) ! set precision
   INTEGER, INTENT( IN ) :: ielemn, nelvar, ninvar, ieltyp, lw1, lw2
   LOGICAL, INTENT( IN ) :: transp
   REAL ( KIND = wp ), INTENT( IN ), DIMENSION ( lw1 ) :: W1
   REAL ( KIND = wp ), INTENT( OUT ), DIMENSION ( lw2 ) :: W2
   SELECT CASE ( ieltyp )
   CASE ( 1 ) ! Element type 1 has a non-trivial transformation
     IF ( transp ) THEN
       W2( 1 ) = W1( 1 )
       W2( 2 ) = W1( 2 )
       W2( 3 ) = W1( 2 )
     ELSE
       W2( 1 ) = W1( 1 )
       W2( 2 ) = W1( 2 ) + W1( 3 )
    END IF
   CASE DEFAULT ! Element 2 has a trivial transformation - no action required
   END SELECT
   RETURN
   END SUBROUTINE RANGE
\end{verbatim} } }
\def\baselinestretch{1.0}

This completes the supply of information to \solver.  The problem may
now be solved using the following program.  We choose to solve the model
subproblem using a preconditioned conjugate gradient scheme, using
diagonal preconditioning, a ``box'' shaped trust-region, calculating an
exact Cauchy point at each iteration but not obtaining an accurate
minimizer of the model within the box. Furthermore, we start from the
initial estimate $x_1 = 0.0$, $x_2 = 0.0$ and $x_3 = 1.5$. As we have no
idea of a good estimate for the Lagrange multiplier associated with the
constraint (group 6), we set $y_6 = 0$.  We observe that the objective
function is bounded below by -2.0 within the feasible box and terminate
when the norms of the reduced gradient of the Lagrangian function and
constraints are smaller than $10^{-5}$.

\def\baselinestretch{0.8}
{%\footnotesize
{\tt \begin{verbatim}
   PROGRAM LANCELOT_example
   USE LANCELOT_double                       ! double precision version
   IMPLICIT NONE
   INTEGER, PARAMETER :: wp = KIND( 1.0D+0 ) ! set precision
   REAL ( KIND = wp ), PARAMETER :: infinity = 10.0_wp ** 20
   TYPE ( LANCELOT_control_type ) :: control
   TYPE ( LANCELOT_inform_type ) :: info
   TYPE ( LANCELOT_data_type ) :: data
   TYPE ( LANCELOT_problem_type ) :: prob
   INTEGER :: i, lfuval
   INTEGER, PARAMETER :: n = 3, ng = 6, nel = 3, nnza = 4, nvrels = 7
   INTEGER, PARAMETER :: ntotel = 3, ngpvlu = 0, nepvlu = 0
   INTEGER :: IVAR( n ), ICALCF( nel ), ICALCG( ng )
   REAL ( KIND = wp ) :: Q( n ), XT( n ), DGRAD( n ), FVALUE( ng )
   REAL ( KIND = wp ) :: GVALUE( ng, 3 )
   REAL ( KIND = wp ), ALLOCATABLE, DIMENSION( : ) :: FUVALS
   EXTERNAL RANGE, ELFUN, GROUP
! make space for problem data
   prob%n = n ; prob%ng = ng ; prob%nel = nel
   ALLOCATE( prob%ISTADG( prob%ng + 1 ), prob%ISTGPA( prob%ng + 1 ) )
   ALLOCATE( prob%ISTADA( prob%ng + 1 ), prob%ISTAEV( prob%nel + 1 ) )
   ALLOCATE( prob%ISTEPA( prob%nel + 1 ), prob%ITYPEG( prob%ng ) )
   ALLOCATE( prob%KNDOFG( prob%ng ), prob%ITYPEE( prob%nel ) )
   ALLOCATE( prob%INTVAR( prob%nel + 1 ) )
   ALLOCATE( prob%IELING( ntotel ), prob%IELVAR( nvrels ), prob%ICNA( nnza ) )
   ALLOCATE( prob%ISTADH( prob%nel + 1 ), prob%A( nnza ) )
   ALLOCATE( prob%B( prob%ng ), prob%BL( prob%n ), prob%BU( prob%n ) )
   ALLOCATE( prob%X( prob%n ), prob%Y( prob%ng ), prob%C( prob%ng ) )
   ALLOCATE( prob%GPVALU( ngpvlu ), prob%EPVALU( nepvlu ) )
   ALLOCATE( prob%ESCALE( ntotel ), prob%GSCALE( prob%ng ) )
   ALLOCATE( prob%VSCALE( prob%n ) )
   ALLOCATE( prob%INTREP( prob%nel ), prob%GXEQX( prob%ng ) )
   ALLOCATE( prob%GNAMES( prob%ng ), prob%VNAMES( prob%n ) )
! set problem data
   prob%ISTADG = (/ 1, 1, 2, 3, 3, 4, 4 /)
   prob%IELVAR = (/ 2, 1, 3, 2, 3, 1, 2 /)
   prob%ISTAEV = (/ 1, 4, 6, 8 /) ; prob%INTVAR( : nel ) = (/ 2,  2, 2 /)
   prob%IELING = (/ 1, 2, 3 /) ; prob%ICNA = (/ 1, 2, 1, 2 /)
   prob%ISTADA = (/ 1, 2, 2, 2, 3, 3, 5 /)
   prob%KNDOFG = (/ 1, 1, 1, 1, 1, 2 /)
   prob%ITYPEG = (/ 1, 0, 2, 0, 1, 3 /) ; prob%ITYPEE = (/ 1, 2, 2 /)
   prob%ISTGPA = (/ 0, 0, 0, 0, 0, 0 /) ; prob%ISTEPA = (/ 0, 0, 0 /)
   prob%A = (/ 1.0_wp, 1.0_wp, 1.0_wp, 2.0_wp /)
   prob%B = (/ 0.0_wp, 0.0_wp, 0.0_wp, 0.0_wp, 0.0_wp, 1.0_wp /)
   prob%BL = (/ - infinity, -1.0_wp, 1.0_wp /)
   prob%BU = (/ infinity, 1.0_wp, 2.0_wp /)
   prob%GSCALE = (/ 1.0_wp, 1.0_wp, 3.0_wp, 1.0_wp, 2.0_wp, 1.0_wp /)
   prob%ESCALE = (/ 1.0_wp, 1.0_wp, 1.0_wp /)
   prob%VSCALE = (/ 1.0_wp, 1.0_wp, 1.0_wp /)
   prob%X = (/ 0.0_wp, 0.0_wp, 1.5_wp /) ;  prob%Y( 6 ) = 0.0_wp
   prob%INTREP = (/ .TRUE., .FALSE., .FALSE. /)
   prob%GXEQX = (/ .FALSE., .TRUE., .FALSE., .TRUE., .FALSE., .FALSE. /)
   prob%GNAMES = (/ 'Obj 1     ', 'Obj 2     ', 'Obj 3     ', 'Obj 4     ', &
                    'Obj 5     ', 'Constraint' /)
   prob%VNAMES = (/ 'x 1', 'x 2', 'x 3' /)
! allocate space for FUVALS
   lfuval = prob%nel + 2 * prob%n
   DO i = 1, prob%nel
     lfuval = lfuval + ( prob%INTVAR( i ) * ( prob%INTVAR( i ) + 3 ) ) / 2
   END DO
   ALLOCATE( FUVALS( lfuval ) )
! problem data complete
   CALL LANCELOT_initialize( data, control )     ! Initialize control parameters
   control%maxit = 100 ; control%out = 6 ! ; control%print_level = 1
   control%stopg = 0.00001_wp ; control%stopc = 0.00001_wp
   control%linear_solver = 1
   control%exact_gcp = .FALSE.
   info%status = 0
! solve the problem
   CALL LANCELOT_solve(                                                 &
       prob, RANGE, GVALUE, FVALUE, XT, FUVALS, lfuval, ICALCF, ICALCG, &
       IVAR, Q, DGRAD, control, info, data, ELFUN  = ELFUN , GROUP = GROUP )
! act on return status
   IF ( info%status == 0 ) THEN                  !  Successful return
     WRITE( 6, "( I6, ' iterations. Optimal objective value =', &
    &       ES12.4, /, ' Optimal solution = ', ( 5ES12.4 ) )" ) &
     info%iter, info%obj, prob%X
   ELSE                                          !  Error returns
     WRITE( 6, "( ' LANCELOT_solve exit status = ', I6 ) " ) info%status
   END IF
   CALL LANCELOT_terminate( data, control, info ) !  delete internal workspace
   DEALLOCATE( prob%GNAMES, prob%VNAMES )         !  delete problem space
   DEALLOCATE( prob%VSCALE, prob%ESCALE, prob%GSCALE, prob%INTREP )
   DEALLOCATE( prob%GPVALU, prob%EPVALU, prob%ITYPEG, prob%GXEQX )
   DEALLOCATE( prob%X, prob%Y, prob%C, prob%A, prob%B, prob%BL, prob%BU )
   DEALLOCATE( prob%ISTADH, prob%IELING, prob%IELVAR, prob%ICNA )
   DEALLOCATE( prob%INTVAR, prob%KNDOFG, prob%ITYPEE, prob%ISTEPA )
   DEALLOCATE( prob%ISTADA, prob%ISTAEV, prob%ISTADG, prob%ISTGPA )
   DEALLOCATE( FUVALS )
   END PROGRAM LANCELOT_example

\end{verbatim} } }
\def\baselinestretch{1.0}

\noindent
This produces the following output.

\def\baselinestretch{0.8}
{%\scriptsize
{\tt \begin{verbatim}
    14 iterations. Optimal objective value = -6.3129E-01
 Optimal solution =   2.4402E-01 -4.0741E-01  1.0000E+00
\end{verbatim} } }
\def\baselinestretch{1.0}

The same effect may be achieved by returning to the user to evaluate
element and group values/derivatives. Simply replace the above call to
\solver\ by the following do-loop:

\def\baselinestretch{0.8}
{%\scriptsize
{\tt \begin{verbatim}
   DO
     CALL LANCELOT_solve(                                                 &
         prob, RANGE, GVALUE, FVALUE, XT, FUVALS, lfuval, ICALCF, ICALCG, &
         IVAR, Q, DGRAD , control, info, data )
     IF ( info%status >= 0 ) EXIT
     IF ( info%status == - 1 .OR. info%status == - 3 .OR. info%status == - 7 ) &
       THEN
       CALL ELFUN ( FUVALS, XT, prob%EPVALU, info%ncalcf, prob%ITYPEE,         &
           prob%ISTAEV, prob%IELVAR, prob%INTVAR, prob%ISTADH, prob%ISTEPA,    &
           ICALCF, prob%nel, prob%nel + 1, prob%ISTAEV( prob%nel + 1 ) - 1,    &
           prob%nel + 1, prob%nel + 1, prob%nel + 1, prob%nel, lfuval, prob%n, &
           prob%ISTEPA( prob%nel + 1 ) - 1, 1, i )
       IF ( i /= 0 ) THEN ; info%status = - 11 ; CYCLE ; END IF
     END IF
     IF ( info%status == - 1 .OR. info%status == - 5 .OR. info%status == - 6 ) &
       THEN
       CALL ELFUN ( FUVALS, XT, prob%EPVALU, info%ncalcf, prob%ITYPEE,         &
           prob%ISTAEV, prob%IELVAR, prob%INTVAR, prob%ISTADH, prob%ISTEPA,    &
           ICALCF, prob%nel, prob%nel + 1, prob%ISTAEV( prob%nel + 1 ) - 1,    &
           prob%nel + 1, prob%nel + 1, prob%nel + 1, prob%nel, lfuval, prob%n, &
           prob%ISTEPA( prob%nel + 1 ) - 1, 3, i )
       IF ( i /= 0 ) THEN ; info%status = - 11 ; CYCLE ; END IF
     END IF
     IF ( info%status == - 2 .OR. info%status == - 4 ) THEN
       CALL GROUP ( GVALUE, prob%ng, FVALUE, prob%GPVALU, info%ncalcg,      &
           prob%ITYPEG, prob%ISTGPA, ICALCG, prob%ng, prob%ng + 1, prob%ng, &
           prob%ng, prob%ISTGPA( prob%ng + 1 ) - 1, .FALSE., i )
       IF ( i /= 0 ) THEN ; info%status = - 11 ; CYCLE ; END IF
     END IF
     IF ( info%status == - 2 .OR. info%status == - 5 ) THEN
       CALL GROUP( GVALUE, prob%ng, FVALUE, prob%GPVALU, info%ncalcg,       &
           prob%ITYPEG, prob%ISTGPA, ICALCG, prob%ng, prob%ng + 1, prob%ng, &
           prob%ng, prob%ISTGPA( prob%ng + 1 ) - 1, .TRUE., i )
       IF ( i /= 0 ) THEN ; info%status = - 11 ; CYCLE ; END IF
     END IF
   END DO
\end{verbatim} } }
\def\baselinestretch{1.0}

If the user (for whatever reason) wishes to be more flexible with
the derivatives used, the optional argument {\tt ELDERS} must be
included and filled appropriately. For example, suppose that we do
not want to provide second derivatives for any elements of type 2
(but to use the PSB approximation instead)
and no derivatives at all for those of type 1 (but finite-difference
gradients and DFP approximate second derivatives).
Then (refering back to {\tt ITYPEE}, we shoud set {\tt ELDERS} as
\begin{center}
{\tt \begin{tabular}{cccc}
i        & 1 & 2 & 3 \\
ELDERS(1,i) & 1 & 0 & 0 \\
ELDERS(2,i) & 2 & 3 & 3
\end{tabular}}
\end{center}
In this case, {\tt ELFUN} should also be
replaced by a subroutine {\tt ELFUN\_flexible} of the following form:

\def\baselinestretch{0.8}
{%\footnotesize
{\tt \begin{verbatim}
   SUBROUTINE ELFUN_flexible( FUVALS, XVALUE, EPVALU, ncalcf, ITYPEE, ISTAEV, &
                              IELVAR, INTVAR, ISTADH, ISTEPA, ICALCF, ltypee, &
                              lstaev, lelvar, lntvar, lstadh, lstepa, lcalcf, &
                              lfuval, lxvalu, lepvlu, llders, ifflag, ELDERS, &
                              ifstat )
   INTEGER, PARAMETER :: wp = KIND( 1.0D+0 ) ! set precision
   INTEGER, INTENT( IN ) :: ncalcf, ifflag, ltypee, lstaev, lelvar, lntvar
   INTEGER, INTENT( IN ) :: lstadh, lstepa, lcalcf, lfuval, lxvalu, lepvlu
   INTEGER, INTENT( IN ) :: llders
   INTEGER, INTENT( OUT ) :: ifstat
   INTEGER, INTENT( IN ) :: ITYPEE( LTYPEE ), ISTAEV( LSTAEV ), IELVAR( LELVAR )
   INTEGER, INTENT( IN ) :: INTVAR( LNTVAR ), ISTADH( LSTADH ), ISTEPA( LSTEPA )
   INTEGER, INTENT( IN ) :: ICALCF( LCALCF ), ELDERS( 2, llders )
   REAL ( KIND = wp ), INTENT( IN ) :: XVALUE( LXVALU )
   REAL ( KIND = wp ), INTENT( IN ) :: EPVALU( LEPVLU )
   REAL ( KIND = wp ), INTENT( INOUT ) :: FUVALS( LFUVAL )
   INTEGER :: ielemn, ieltyp, ihstrt, ilstrt, igstrt, ipstrt, jcalcf
   REAL ( KIND = wp ) :: v1, v2, u1, u2, u3, cs, sn
   ifstat = 0
   DO jcalcf = 1, ncalcf
     ielemn = ICALCF( jcalcf )
     ilstrt = ISTAEV( ielemn ) - 1
     igstrt = INTVAR( ielemn ) - 1
     ipstrt = ISTEPA( ielemn ) - 1
     IF ( ELDERS( 2, ielemn ) <= 0 ) ihstrt = ISTADH( ielemn ) - 1
     ieltyp = ITYPEE( ielemn )
     SELECT CASE ( ieltyp )
     CASE ( 1 )
       u1 = XVALUE( IELVAR( ilstrt + 1 ) )
       u2 = XVALUE( IELVAR( ilstrt + 2 ) )
       u3 = XVALUE( IELVAR( ilstrt + 3 ) )
       v1 = u1
       v2 = u2 + u3
       cs = COS( v2 )
       sn = SIN( v2 )
       IF ( ifflag == 1 ) THEN
         FUVALS( ielemn ) = v1 * sn
       ELSE
! no first or second derivatives for element of type 1
       END IF
     CASE ( 2 )
       u1 = XVALUE( IELVAR( ilstrt + 1 ) )
       u2 = XVALUE( IELVAR( ilstrt + 2 ) )
       IF ( ifflag == 1 ) THEN
         FUVALS( ielemn ) = u1 * u2
       ELSE
         IF ( ELDERS( 1, ielemn ) <= 0 ) THEN
           FUVALS( igstrt + 1 ) = u2
           FUVALS( igstrt + 2 ) = u1
! no second derivatives for element of type 2
         END IF
       END IF
     END SELECT
   END DO
   RETURN
   END SUBROUTINE ELFUN_flexible
\end{verbatim} } }
\def\baselinestretch{1.0}

\end{document}




