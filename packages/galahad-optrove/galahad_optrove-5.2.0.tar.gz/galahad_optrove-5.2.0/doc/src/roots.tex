\documentclass{galahad}

% set the package name

\newcommand{\package}{roots}
\newcommand{\packagename}{ROOTS}
\newcommand{\fullpackagename}{\libraryname\_\packagename}

\begin{document}

\galheader

%%%%%%%%%%%%%%%%%%%%%% SUMMARY %%%%%%%%%%%%%%%%%%%%%%%%

\galsummary
This package uses classical formulae together with Newton's 
method to find all the real roots of a real polynomial.

%%%%%%%%%%%%%%%%%%%%%% attributes %%%%%%%%%%%%%%%%%%%%%%%%

\galattributes
\galversions{\tt  \fullpackagename\_single, \fullpackagename\_double}.
\galuses
{\tt GALAHAD\_SY\-M\-BOLS}, 
{\tt GALAHAD\-\_SPACE}, 
{\tt GALAHAD\_SPECFILE},
{\tt GALAHAD\_SORT}.
\galdate November 2010.
\galorigin N. I. M. Gould, Rutherford Appleton Laboratory.
\gallanguage Fortran~95 + TR 15581 or Fortran~2003. 

%%%%%%%%%%%%%%%%%%%%%% HOW TO USE %%%%%%%%%%%%%%%%%%%%%%%%

\galhowto

The package is available with single, double and (if available) 
quadruple precision reals, and either 32-bit or 64-bit integers. 
Access to the 32-bit integer,
single precision version requires the {\tt USE} statement
\medskip

\hspace{8mm} {\tt USE \fullpackagename\_single}

\medskip
\noindent
with the obvious substitution 
{\tt \fullpackagename\_double},
{\tt \fullpackagename\_quadruple},
{\tt \fullpackagename\_single\_64},
{\tt \fullpackagename\_double\_64} and
{\tt \fullpackagename\_quadruple\_64} for the other variants.


\noindent
If it is required to use more than one of the modules at the same time, 
the derived types
{\tt \packagename\_control\_type}, 
{\tt \packagename\_inform\_type} 
and
{\tt \packagename\_data\_type}
(Section~\ref{galtypes})
and the subroutine
{\tt \packagename\_\-solve},
(Section~\ref{galarguments})
must be renamed on one of the {\tt USE} statements.

%%%%%%%%%%%%%%%%%%%%%%%%%%% kinds %%%%%%%%%%%%%%%%%%%%%%%%%%%

%%%%%%%%%%%%%%%%%%%%%% kinds %%%%%%%%%%%%%%%%%%%%%%

\galkinds
We use the terms integer and real to refer to the fortran
keywords {\tt REAL(rp\_)} and {\tt INTEGER(ip\_)}, where 
{\tt rp\_} and {\tt ip\_} are the relevant kind values for the
real and integer types employed by the particular module in use. 
The former are equivalent to
default {\tt REAL} for the single precision versions,
{\tt DOUBLE PRECISION} for the double precision cases
and quadruple-precision if 128-bit reals are available, and 
correspond to {\tt rp\_ = real32}, {\tt rp\_ = real64} and
{\tt rp\_ = real128} respectively as defined by the 
fortran {\tt iso\_fortran\_env} module.
The latter are default (32-bit) and long (64-bit) integers, and
correspond to {\tt ip\_ = int32} and {\tt ip\_ = int64},
respectively, again from the {\tt iso\_fortran\_env} module.


%%%%%%%%%%%%%%%%%%%%%% derived types %%%%%%%%%%%%%%%%%%%%%%%%

\galtypes
Three derived data types are accessible from the package.

%%%%%%%%%%% control type %%%%%%%%%%%

\subsubsection{The derived data type for holding control 
 parameters}\label{typecontrol}
The derived data type 
{\tt \packagename\_control\_type} 
is used to hold controlling data. Default values may be obtained by calling 
{\tt \packagename\_initialize}
(see Section~\ref{subinit}),
while components may also be changed by calling 
{\tt \packagename\_read\-\_specfile}
(see Section~\ref{readspec}). 
The components of 
{\tt \packagename\_control\_type} 
are:

\begin{description}

\itt{error} is a scalar variable of type \integer, that holds the
stream number for error messages. Printing of error messages in 
{\tt \packagename\_solve} and {\tt \packagename\_terminate} is suppressed if 
{\tt error} $\leq 0$.
The default is {\tt error = 6}.

\ittf{out} is a scalar variable of type \integer, that holds the
stream number for informational messages. Printing of informational messages in 
{\tt \packagename\_solve} is suppressed if {\tt out} $< 0$.
The default is {\tt out = 6}.

\itt{print\_level} is a scalar variable of type \integer, that is used
to control the amount of informational output which is required. No 
informational output will occur if {\tt print\_level} $\leq 0$. If 
{\tt print\_level} $\geq 1$, debugging information will be provided.
The default is {\tt print\_level = 0}.

\ittf{tol} is an \intentin\ scalar of type \realdp\, that should
be set to the required accuracy of the roots. Every effort
will be taken to ensure that each computed root $x_c$ lies within
$\pm$ {\tt tol} $x_e$ of 
its exact equivalent $x_e$, although sometimes the required accuracy will not
be possible.
The default is {\tt tol = EPSILON(1.0)} ({\tt EPSILON(1.0D0)} in 
{\tt \fullpackagename\_double}).

\itt{space\_critical} is a scalar variable of type default \logical, 
that must be set \true\ if space is critical when allocating arrays
and  \false\ otherwise. The package may run faster if 
{\tt space\_critical} is \false\ but at the possible expense of a larger
storage requirement. The default is {\tt space\_critical = .FALSE.}.

\itt{deallocate\_error\_fatal} is a scalar variable of type default \logical, 
that must be set \true\ if the user wishes to terminate execution if
a deallocation  fails, and \false\ if an attempt to continue
will be made. The default is {\tt deallocate\_error\_fatal = .FALSE.}.

\itt{prefix} is a scalar variable of type default \character\
and length 30, that may be used to provide a user-selected 
character string to preface every line of printed output. 
Specifically, each line of output will be prefaced by the string 
{\tt prefix(2:LEN(TRIM(prefix))-1)},
thus ignoring the first and last non-null components of the
supplied string. If the user does not want to preface lines by such
a string, they may use the default {\tt prefix = ""}.

\end{description}

%%%%%%%%%%% inform type %%%%%%%%%%%

\subsubsection{The derived data type for holding informational
 parameters}\label{typeinform}
The derived data type 
{\tt \packagename\_inform\_type} 
is used to hold parameters that give information about the progress and needs 
of the algorithm. The components of 
{\tt \packagename\_inform\_type} 
are:

\begin{description}

\itt{status} is a scalar variable of type \integer, that gives the
exit status of the algorithm. 
%See Sections~\ref{galerrors} and \ref{galinfo}
See Section~\ref{galerrors} 
for details.

\itt{alloc\_status} is a scalar variable of type \integer, that gives
the status of the last attempted array allocation or deallocation.
This will be 0 if {\tt status = 0}.

\itt{bad\_alloc} is a scalar variable of type default \character\
and length 80, that  gives the name of the last internal array 
for which there were allocation or deallocation errors.
This will be the null string if {\tt status = 0}. 

\end{description}

%%%%%%%%%%% data type %%%%%%%%%%%

\subsubsection{The derived data type for holding problem data}\label{typedata}
The derived data type 
{\tt \packagename\_data\_type} 
is used to hold all the data for a particular problem,
between calls of 
{\tt \packagename} procedures. 
This data should be preserved, untouched, from the initial call to 
{\tt \packagename\_initialize}
to the final call to
{\tt \packagename\_terminate}.

%%%%%%%%%%%%%%%%%%%%%% argument lists %%%%%%%%%%%%%%%%%%%%%%%%

\galarguments
There are three procedures for user calls
(see Section~\ref{galfeatures} for further features): 
\begin{enumerate}
\item The subroutine 
      {\tt \packagename\_initialize} 
      is used to set default values, and initialize private data.
\item The subroutine 
      {\tt \packagename\_solve} 
      is called to find the real roots of the polynomial 
\eqn{poly}{\sum_{i=0}^d a_i x^i}
      of degree $d$, where the coefficients $a_i$, $0 \leq i \leq d$ are real.
\item The subroutine 
      {\tt \packagename\_terminate} 
      allows the user to automatically deallocate array 
       components of the private data, allocated by 
       {\tt \packagename\_solve}, 
       at the end of the solution process. 
\end{enumerate}

%%%%%% initialization subroutine %%%%%%

\subsubsection{The initialization subroutine}\label{subinit}
 Default values are provided as follows:
\vspace*{1mm}

\hspace{8mm}
{\tt CALL \packagename\_initialize( data, control, inform )}

\vspace*{-3mm}
\begin{description}

\itt{data} is a scalar \intentinout\ argument of type 
{\tt \packagename\_data\_type}
(see Section~\ref{typedata}). It is used to hold data about the problem being 
solved. 

\itt{control} is a scalar \intentout\ argument of type 
{\tt \packagename\_control\_type}
(see Section~\ref{typecontrol}). 
On exit, {\tt control} contains default values for the components as
described in Section~\ref{typecontrol}.
These values should only be changed after calling 
{\tt \packagename\_initialize}.

\itt{inform} is a scalar \intentout\ argument of type 
{\tt \packagename\_inform\_type}
(see Section~\ref{typeinform}). A successful call to
{\tt \packagename\_initialize}
is indicated when the  component {\tt status} has the value 0. 
%For other return values of {\tt status}, see Section~\ref{galerrors}.

\end{description}


%%%%%%%%% main solution subroutine %%%%%%

\subsubsection{The solution subroutine}
The roots of the polynomial \req{poly} are found as follows
\vspace*{1mm}

\hspace{8mm}
{\tt CALL \packagename\_solve( A, nroots, ROOTS, control, inform, data )}

\vspace*{-1mm}
\begin{description}
\ittf{A} is an \intentin\ rank-one array of type \realdp, whose 
lower bound must be {\tt 0} and whose upper bound specifies the degree, $d$,
of the polynomial. The entries {\tt A}$(i), i = 0$, \ldots, {\tt UBOUND(A)},
must contain the values of the real coefficients $a_i$, $0 \leq i \leq d$.
\restrictions {\tt UBOUND(A,1)} $\geq 0$.

\itt{nroots} is an \intentout\ scalar of type \integer,
that gives the number of real roots of the polynomial.

\itt{ROOTS} is an \intentout\ rank-one array of length $d$ and 
type \realdp. On exit, {\tt ROOTS(:nroots)} give the values
of the real roots of the polynomial in increasing order.
\restrictions {\tt SIZE(ROOTS)} $\geq$ {\tt UBOUND(A,1)}.

\itt{control} is a scalar \intentin\ argument of type 
{\tt \packagename\_control\_type}
(see Section~\ref{typecontrol}). Default values may be assigned by calling 
{\tt \packagename\_initialize} prior to the first call to 
{\tt \packagename\_solve}.

\itt{inform} is a scalar \intentinout\ argument of type 
{\tt \packagename\_inform\_type}
(see Section~\ref{typeinform}). 
A successful call to
{\tt \packagename\_solve}
is indicated when the  component {\tt status} has the value 0. 
For other return values of {\tt status}, see Section~\ref{galerrors}.

\itt{data} is a scalar \intentinout\ argument of type 
{\tt \packagename\_data\_type}
(see Section~\ref{typedata}). It is used to hold data about the problem being 
solved. It must not have been altered {\bf by the user} since the last call to 
{\tt \packagename\_initialize}.

\end{description}

%%%%%%% termination subroutine %%%%%%

\subsubsection{The  termination subroutine}
All previously allocated arrays are deallocated as follows:
\vspace*{1mm}

\hspace{8mm}
{\tt CALL \packagename\_terminate( data, control, inform )}

\vspace*{-3mm}
\begin{description}

\itt{data} is a scalar \intentinout\ argument of type 
{\tt \packagename\_data\_type} 
exactly as for
{\tt \packagename\_solve},
which must not have been altered {\bf by the user} since the last call to 
{\tt \packagename\_initialize}.
On exit, array components will have been deallocated.

\itt{control} is a scalar \intentin\ argument of type 
{\tt \packagename\_control\_type}
exactly as for
{\tt \packagename\_solve}.

\itt{inform} is a scalar \intentout\ argument of type
{\tt \packagename\_inform\_type}
exactly as for
{\tt \packagename\_solve}.
Only the component {\tt status} will be set on exit, and a 
successful call to 
{\tt \packagename\_terminate}
is indicated when this  component {\tt status} has the value 0. 
For other return values of {\tt status}, see Section~\ref{galerrors}.

\end{description}

%%%%%%%%%%%%%%%%%%%%%% Warning and error messages %%%%%%%%%%%%%%%%%%%%%%%%

\galerrors
A negative value of {\tt inform\%status} on exit from 
{\tt \packagename\_solve}
indicates that an error has occurred. No further calls should be made
until the error has been corrected. Possible values are:

\begin{description}

\itt{\galerrallocate.} An allocation error occured. A message indicating 
the offending 
array is written on unit {\tt control\%error}, and the returned allocation 
status and a string containing the name of the offending array
are held in {\tt inform\%alloc\_\-status}
and {\tt inform\%bad\_alloc} respectively.

\itt{\galerrdeallocate.} A deallocation error occured. 
A message indicating the offending 
array is written on unit {\tt control\%error} and the returned allocation 
status and a string containing the name of the offending array
are held in {\tt inform\%alloc\_\-status}
and {\tt inform\%bad\_alloc} respectively.

\itt{\galerrrestrictions.} 
Either the specified degree of the polynomial in {\tt degree} is less than 0, 
or the declared dimension of the array {\tt ROOTS} is smaller
than the specified degree.

\end{description}

%%%%%%%%%%%%%%%%%%%%%% Further features %%%%%%%%%%%%%%%%%%%%%%%%

\galfeatures
\noindent In this section, we describe an alternative means of setting 
control parameters, that is components of the variable {\tt control} of type
{\tt \packagename\_control\_type}
(see Section~\ref{typecontrol}), 
by reading an appropriate data specification file using the
subroutine {\tt \packagename\_read\_specfile}. This facility
is useful as it allows a user to change  {\tt \packagename} control parameters 
without editing and recompiling programs that call {\tt \packagename}.

A specification file, or specfile, is a data file containing a number of 
"specification commands". Each command occurs on a separate line, 
and comprises a "keyword", 
which is a string (in a close-to-natural language) used to identify a 
control parameter, and 
an (optional) "value", which defines the value to be assigned to the given
control parameter. All keywords and values are case insensitive, 
keywords may be preceded by one or more blanks but
values must not contain blanks, and
each value must be separated from its keyword by at least one blank.
Values must not contain more than 30 characters, and 
each line of the specfile is limited to 80 characters,
including the blanks separating keyword and value.

The portion of the specification file used by 
{\tt \packagename\_read\_specfile}
must start
with a "{\tt BEGIN \packagename}" command and end with an 
"{\tt END}" command.  The syntax of the specfile is thus defined as follows:
\begin{verbatim}
  ( .. lines ignored by CQP_read_specfile .. )
    BEGIN CQP
       keyword    value
       .......    .....
       keyword    value
    END 
  ( .. lines ignored by CQP_read_specfile .. )
\end{verbatim}
where keyword and value are two strings separated by (at least) one blank.
The ``{\tt BEGIN \packagename}'' and ``{\tt END}'' delimiter command lines 
may contain additional (trailing) strings so long as such strings are 
separated by one or more blanks, so that lines such as
\begin{verbatim}
    BEGIN CQP SPECIFICATION
\end{verbatim}
and
\begin{verbatim}
    END CQP SPECIFICATION
\end{verbatim}
are acceptable. Furthermore, 
between the
``{\tt BEGIN \packagename}'' and ``{\tt END}'' delimiters,
specification commands may occur in any order.  Blank lines and
lines whose first non-blank character is {\tt !} or {\tt *} are ignored. 
The content 
of a line after a {\tt !} or {\tt *} character is also 
ignored (as is the {\tt !} or {\tt *}
character itself). This provides an easy manner to "comment out" some 
specification commands, or to comment specific values 
of certain control parameters.  

The value of a control parameters may be of three different types, namely
integer, logical or real.
Integer and real values may be expressed in any relevant Fortran integer and
floating-point formats (respectively). Permitted values for logical
parameters are "{\tt ON}", "{\tt TRUE}", "{\tt .TRUE.}", "{\tt T}", 
"{\tt YES}", "{\tt Y}", or "{\tt OFF}", "{\tt NO}",
"{\tt N}", "{\tt FALSE}", "{\tt .FALSE.}" and "{\tt F}". 
Empty values are also allowed for 
logical control parameters, and are interpreted as "{\tt TRUE}".  

The specification file must be open for 
input when {\tt \packagename\_read\_specfile}
is called, and the associated device number 
passed to the routine in device (see below). 
Note that the corresponding 
file is {\tt REWIND}ed, which makes it possible to combine the specifications 
for more than one program/routine.  For the same reason, the file is not
closed by {\tt \packagename\_read\_specfile}.

\subsubsection{To read control parameters from a specification file}
\label{readspec}

Control parameters may be read from a file as follows:
\hskip0.5in 

\def\baselinestretch{0.8}
{\tt 
\begin{verbatim}
     CALL CQP_read_specfile( control, device )
\end{verbatim}
}
\def\baselinestretch{1.0}

\begin{description}
\itt{control} is a scalar \intentinout argument of type 
{\tt \packagename\_control\_type}
(see Section~\ref{typecontrol}). 
Default values should have already been set, perhaps by calling 
{\tt \packagename\_initialize}.
On exit, individual components of {\tt control} may have been changed
according to the commands found in the specfile. Specfile commands and 
the component (see Section~\ref{typecontrol}) of {\tt control} 
that each affects are given in Table~\ref{specfile}.

\bctable{|l|l|l|} 
\hline
  command & component of {\tt control} & value type \\ 
\hline
  {\tt error-printout-device} & {\tt \%error} & integer \\
  {\tt printout-device} & {\tt \%out} & integer \\
  {\tt print-level} & {\tt \%print\_level} & integer \\
  {\tt root-tolerance} & {\tt \%tol} & real \\
  {\tt space-critical}   & {\tt \%space\_critical} & logical \\
  {\tt deallocate-error-fatal}   & {\tt \%deallocate\_error\_fatal} & logical \\
  {\tt output-line-prefix} & {\tt \%prefix} & {\tt character} \\
\hline

\ectable{\label{specfile}Specfile commands and associated 
components of {\tt control}.}

\itt{device} is a scalar \intentin argument of type \integer,
that must be set to the unit number on which the specfile
has been opened. If {\tt device} is not open, {\tt control} will
not be altered and execution will continue, but an error message
will be printed on unit {\tt control\%error}.

\end{description}

%%%%%%%%%%%%%%%%%%%%%% GENERAL INFORMATION %%%%%%%%%%%%%%%%%%%%%%%%

\galgeneral

\galcommon None.
\galworkspace Provided automatically by the module.
\galroutines None. 
\galmodules {\tt \packagename\_solve} calls the \galahad\ packages
{\tt GALAHAD\_SY\-M\-BOLS}, 
{\tt GALAHAD\_SPACE},
{\tt GALAHAD\_SPECFILE}
and
{\tt GAL\-AHAD\_SORT}.
\galio Output is under control of the arguments
 {\tt control\%error}, {\tt control\%out} and {\tt control\%print\_level}.
\galportability ISO Fortran~95 + TR 15581 or Fortran~2003. 
The package is thread-safe.

%%%%%%%%%%%%%%%%%%%%%% METHOD %%%%%%%%%%%%%%%%%%%%%%%%

\galmethod
Littlewood and Ferrari's algorithms are used to find estimates of the
real roots of cubic and quartic polynomials, respectively; a stabilized version 
of the well-known formula is used in the quadratic case. Newton's
method and/or methods based on the companion matrix are used to further 
refine the computed roots if necessary. Madsen and Reid's (1975) method 
is used for polynomials whose degree exceeds four.
\vspace*{1mm}

\galreferences
\vspace*{1mm}

\noindent
The basic method is that given by
\vspace*{1mm}

\noindent
K. Madsen and J. K. Reid, 
``FORTRAN Subroutines for Finding Polynomial Zeros''.
Technical Report A.E.R.E. R.7986, Computer Science and System Division, 
A.E.R.E. Harwell, Oxfordshire, U.K. (1975)

%%%%%%%%%%%%%%%%%%%%%% EXAMPLE %%%%%%%%%%%%%%%%%%%%%%%%

\galexample
Suppose we wish to solve the quadratic, cubic, quartic and quintic equations
\disp{\arr{rl}{
x^2 - 3 x + 2 \!\!\!\!\!\!\! & = 0 \\
x^3 - 6 x^2 + 11 x - 6 \!\!\!\!\!\!\! & = 0 \\
x^4 - 10 x^3 +35 x^2 - 50 x + 24 \!\!\!\!\!\!\! & = 0 \tim{and} \\
x^5 - 15 x^4 + 85 x^3 - 225 x^2 + 274 x - 120 \!\!\!\!\!\!\! & = 0 .}}
Then we may use the following code:

{\tt \small
\VerbatimInput{\packageexample}
}
\noindent
This produces the following output:
{\tt \small
\VerbatimInput{\packageresults}
}

\end{document}

