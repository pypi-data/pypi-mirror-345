
\Chapter{Introduction}

A number field is a finite extension of the field of rational numbers. 
{\Alnuth} provides various methods to compute with number fields
which are given by a defining polynomial or by generators. For background 
on number fields we refer to \cite{Sta79}.

Some of the methods provided in this package are written in {\GAP} code.
The other part of the methods is imported from the Computer Algebra 
System PARI/GP \cite{PARI2}. Hence this package contains some {\GAP}
functions  and an interface to some functions in the computer algebra
system PARI/GP. Therefore one has to have PARI/GP installed to use the
full functionality of {\Alnuth}.

We note that only a very small part of the functions available in PARI/GP
are linked to {\GAP} and PARI/GP provides many more methods for
computations in number fields. 

The main methods included in {\Alnuth} are: creating a number field, 
computing its maximal order (using PARI/GP), computing its unit group (using 
PARI/GP) and a presentation of this unit group, computing the elements of a
given norm of the number field (using PARI/GP), determining a presentation
for a finitely generated multiplicative subgroup (using PARI/GP), and
factoring polynomials defined over number fields (using PARI/GP). For
background on algorithms for number fields we refer to \cite{Poh93},
\cite{PZa89} and \cite{Coh93}.

The functions provided by {\Alnuth} are introduced in the following
chapter. Then an example application is outlined. In the final chapter
of this manual the installation of the package and configuration of
the interface, including hints on the installation of PARI/GP, are
described.

%%%%%%%%%%%%%%%%%%%%%%%%%%%%%%%%%%%%%%%%%%%%%%%%%%%%%%%%%%%%%%%%%%
\Section{Acknowledgements}

To begin with we are very grateful for all the feedback by users of
former versions of {\Alnuth}.

We thank Bill Allombert who wrote the GP code for the interface to
PARI/GP and who was extremely helpful in the transition from KANT
to PARI/GP.

For feedback on the development version, including a code patch, we
are much obliged to Max Horn.

The second author acknowledges the financial support at CAUL within
the projects PTDC/MAT/101993/2008 and ISFL-1-143, financed by FEDER
and FCT, in the development of Version 3 of {\Alnuth}.

%%%%%%%%%%%%%%%%%%%%%%%%%%%%%%%%%%%%%%%%%%%%%%%%%%%%%%%%%%%%%%%%%%
\Section{License}

This program is free software: you can redistribute it and/or modify
it under the terms of the GNU General Public License as published by
the Free Software Foundation, either version 2 of the license, or
(at your option) any later version.

This program is distributed in the hope that it will be useful,
but WITHOUT ANY WARRANTY; without even the implied warranty of
MERCHANTABILITY or FITNESS FOR A PARTICULAR PURPOSE.  See the
GNU General Public License for more details.

You should have received a copy of the GNU General Public License
along with this program. If not, see \URL{https://www.gnu.org/licenses/}

%%%%%%%%%%%%%%%%%%%%%%%%%%%%%%%%%%%%%%%%%%%%%%%%%%%%%%%%%%%%%%%%%%%%%%%%%
%%
%E
